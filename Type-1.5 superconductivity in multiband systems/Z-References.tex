
\begin{thebibliography}{99}
  \addcontentsline{toc}{chapter}{Ссылки}
  \label{ch:Z}
\bibitem{bib:1} E. Babaev \& J.M. Speight Phys.Rev. B 72 180502 (2005)

\bibitem{bib:2} E. Babaev, J. Carlstrom, J. M. Speight Phys. Rev. Lett. 105, 067003 (2010)

\bibitem{bib:3} A. Abrikosov Sov. Phys. JETP 5, 1174 (1957)

\bibitem{bib:4} L. Landau, Nature (London, United Kingdom) 141, 688 (1938). R. P. Huebener, Magnetic Flux Structures of Superconductors, 2nd ed. (Springer-Verlag, New-York, 2001). R. Prozorov, A. F. Fidler, J. Hoberg, P. C. Canfield, Nature Physics 4, 327 - 332 (2008)

\bibitem{bib:5} The noninteracting regime, which is frequently called “Bogomolny limit” is a property of Ginzburg-Landau model where, at \( \kappa = 1/\sqrt{2} \), the core-core attractive interaction between vortices exactly cancels the current-current repulsive interaction. However, in a realistic system extremely close to the \( \kappa \to 1/\sqrt{2} \) limit one should go beyond the Ginzburg-Landau description to determine leftover inter-vortex interactions, because these are then determined not by the fundamental length scales of the GL theory but by small microscopic effects leading to non-universal weak vortex interaction potentials. We do not consider the physics which arises in the Bogomolny limits in this work.

\bibitem{bib:6} H. Suhl, B. T. Matthias, and L. R. Walker Phys. Rev. Lett. 3, 552 (1959)

\bibitem{bib:7} A. Liu, I.I. Mazin, J. Kortus, Phys. Rev. Lett. 87 087005 (2001); I.I. Mazin, et al., Phys. Rev. Lett. 89 (2002) 107002.

\bibitem{bib:8} A. Gurevich, Phys. Rev. B 67 184515 (2003) ;

\bibitem{bib:9} A. Gurevich, Physica C 056 160 (2007)

\bibitem{bib:10} M. E. Zhitomirsky and V.-H. Dao, Phys. Rev. B69, 054508 (2004).

\bibitem{bib:11} see e.g. K. Ishida, Y. Nakai,H., Hosono, J. Phys. Soc. Jpn. 78 062001 (2009)

\bibitem{bib:12} N.W. Ashcroft, J. Phys. Condens. Matter 12, A129 (2000);Phys. Rev. Lett. 92, 187002 (2004) K. Moulopoulos and N. W. Ashcroft Phys. Rev. Lett. 66 2915 (1991),

\bibitem{bib:13} E. Babaev, A. Sudb\o\,\! and N.W. Ashcroft Nature 431 666 (2004), E. Sm\o\,\!rgrav, J. Smiseth, E. Babaev, A. Sudb\o\,\!Phys. Rev. Lett. 94, 096401 (2005) E. Babaev, N.W. Ashcroft Nature Physics 3, 530 (2007)

\bibitem{bib:14} E. V. Herland, E. Babaev, A. Sudbo Phys. Rev. B 82, 134511 (2010)

\bibitem{bib:15} P. B. Jones, Mon. Not. Royal Astr. Soc. 371, 1327 (2006); E. Babaev, Phys. Rev. Lett. 103, 231101 (2009).

\bibitem{bib:16} V.V. Moshchalkov M Menghini, T. Nishio, Q. H. Chen, A. V. Silhanek, V. H. Dao, L. F. Chibotaru, N. D. Zhigadlo, and J. Karpinski Phys. Rev. Lett. 102, 117001 (2009)

\bibitem{bib:17} T. Nishio, Q. Chen, W. Gillijns, K. De Keyser, K. Vervaeke, and V. V. Moshchalkov Phys. Rev. B 81, 020506(R) (2010)

\bibitem{bib:18} R. Geurts, M. V. Milosevic, F. M. Peeters Phys. Rev. B 81, 214514 (2010) V. H. Dao, L. F. Chibotaru, T. Nishio, V. V. Moshchalkov Phys. Rev. B 83 020503 (2011)

\bibitem{bib:19} J.M. Speight, Phys. Rev. D 55 3830 (1997) .

\bibitem{bib:20} B. Plohr, J. Math. Phys. 22, 2184 (1981)

\bibitem{bib:21} In the case of independently conserved condensates there could be entrainment effects [see e.g. A.F. Andreev and E. Bashkin, Sov. Phys. JETP 42, 164 (1975) leading to different mixed gradient terms which are quadratic in gradients and quartic in \( \psi_a \) ; some of the effects of which on vortex physics were considered in E. Babaev Phys. Rev. B 79, 104506 (2009), and in\cite{bib:14}

\bibitem{bib:22} E. Babaev Phys.Rev.Lett. 89 (2002) 067001 E. Babaev, J. Jaykka, and M. Speight, Phys. Rev. Lett. 103, 237002 (2009); M.A. Silaev Phys. Rev. B 83, 144519 (2011)

\bibitem{bib:23} M. Abramowitz and I.A. Stegun, Handbook of Mathematical Functions (Dover, New York NY, USA, 1972) p. 374.

\bibitem{bib:24} E. Babaev, J. Carlstrom, J.M. Speight not published.

\bibitem{bib:25} M. Silaev and E. Babaev arXiv:1102.5734

\bibitem{bib:26} J.K. Perring and T.H.R. Skyrme, Nucl. Phys. 31 (1962) 550.

\end{thebibliography}