\chapter{Вихревая асимптотика}
\label{ch:3}

The key to understanding the interaction of well separated vortices is to 
analyze the large \( r \) asymptotics of the vortex solution. We will analyze 
this problem in the context of a general TCGL model whose free energy takes 
the form
\begin{equation}
(3)
\end{equation}
where \( F_p \) contains all the non-gradient terms (in particular, but not 
restricted to, Josephson and density-density interaction terms). This free 
energy is consistent with (2) in the case \( \nu = 0 \) We will show in 
section IIID how to handle mixed gradient terms. The precise form of \( F_p \) 
is not crucial for our analysis in this section. By gauge invariance, it can 
depend on the condensates only via \( |\psi_1|, |\psi_2| \) and (if the 
condensates are not independently conserved) on \( \theta_1 - \theta_2 \). We 
will assume that \( F_p \) takes its minimum value (which we normalize to be 
0) when \( |\psi_1| = u_1 > 0, |\psi_2| = u_2 > 0 \) and 
\( \theta_1 - \theta_2 = 0 \). So, either there is no phase coupling 
(\( F_p \) is independent of \( \theta_1 - \theta_2 \)) and the choice of 
\( \theta_1 - \theta_2 = 0 \) is arbitrary, or the phase coupling is such as 
to encourage phase locking. (Note that the case of phase anti-locked fields 
can trivially be recovered from our analysis by mapping 
\( \psi_2 \mapsto -\psi_1 \)).

The field equations are obtained from \( F \) by demanding that the total 
free energy \( E = \int F dx_1 dx_2 \) is stationary with respect to all 
variations of \( \psi_1, \psi_2 \) and \( A_i \). A routine calculation yields 
\begin{equation}
(4)
(5)
\end{equation}
This triple of coupled nonlinear partial differential equations supports 
solutions of the form
\begin{equation}
(6)
\end{equation}
where \( f_1, f_2, a \) are real profile functions. Note that in some cases 
mixed gradient terms favour non-axially symmetric solutions. In this section 
we consider only axially symmetric vortices. Fields within the above ansatz 
satisfy the field equations if and only if the profile functions 
\( f_1(r), f_2(r), a(r) \) satisfy the coupled ordinary differential equation 
system
\begin{equation}
(7)
(8)
\end{equation}
The solution we require, the vortex, has boundary behaviour 
\( f_a(r) \rightarrow u_a \), \( a(r) \rightarrow -1/e \) as 
\( r \rightarrow \infty \). So, for large \( r \), the quantities 
\begin{equation}
(9)
\end{equation}
are small and so should, to leading order, satisfy the linearization of 
(7),(8) about \( (u_1, u_2, -1/e) \). That is, at large \( r \),
\begin{equation}
(10)
(11)
\end{equation}
where \( \mathcal{H} \) is the Hessian matrix of 
\( F_p(|\psi_1|, |\psi_2|, 0) \) about its minimum 
\begin{equation}
(12)
\end{equation}
So \( \alpha \) decouples from \( \epsilon_1, \epsilon_2\) asymptotically, and 
we see immediately that
\begin{equation}
(13)
\end{equation}
where \( K_n \) denotes the nth modified Bessel’s function of the second 
kind\( ^{23} \), and \( q_0 \) is an unknown real constant. Hence, at large,
\begin{equation}
(14)
\end{equation}
Since, for all \( n \), 
\begin{equation}
(15)
\end{equation}
it follows that the magnetic field decays exponentially as a function of 
\( r \), with length scale (penetration depth)
\begin{equation}
(16)
\end{equation}

By contrast, (10) represents, in general, a coupled pair of ordinary 
differential equations for \( \epsilon_1, \epsilon_2 \). Since 
\( (u_1, u_2, 0 ) \) is a \textit{minimum} of 
\( F_p(|\psi_1|, |\psi_2|, \theta_1 - \theta_2) \), the Hessian matrix is a 
\textit{positive definite} symmetric \( 2\times2 \) real matrix. Hence its 
eigenvalues, \( \mu_1^2, \mu_2^2\) say, are real and positive, and its 
eigenvectors, \( v_1, v_2 \) say, form an orthonormal basis for 
\( \mathbb{R} \). Expanding in the basis \( v_1, v_2 \)
\begin{equation}
(17)
\end{equation}

\section{Симметричная модель \texorpdfstring{$ U(1) \times U(1) $}
  {U(1) x U(1)}}
\label{sec:3-1}

\section{Джозефоновский контакт}
\label{sec:3-2}

\subsection{Сравнение со случаем пассивной второй band (?)}
\label{subsec:3-2-1}

\section{Контакт плотность-плотность (?)}
\label{sec:3-3}

\section{Условия смешанного градиента}
\label{sec:3-4}
