\chapter{Вихревая асимптотика}
\label{ch:3}

The key to understanding the interaction of well separated vortices is to 
analyze the large \( r \) asymptotics of the vortex solution. We will analyze 
this problem in the context of a general TCGL model whose free energy takes 
the form
\begin{equation}
    F = \frac{1}{2}\left( D_i \psi_1 \right)^{*} D_i \psi_1 + 
        \frac{1}{2}\left( D_i \psi_2 \right)^{*} D_i \psi_2 + 
        \frac{1}{2}\left( \partial_1 A_2 - \partial_2 A_1 \right)^2 + F_p
    \label{eq:3}
\end{equation}
where \( F_p \) contains all the non-gradient terms (in particular, but not 
restricted to, Josephson and density-density interaction terms). This free 
energy is consistent with \eqref{eq:2} in the case \( \nu = 0 \) We will show in 
section IIID how to handle mixed gradient terms. The precise form of \( F_p \) 
is not crucial for our analysis in this section. By gauge invariance, it can 
depend on the condensates only via \( |\psi_1|, |\psi_2| \) and (if the 
condensates are not independently conserved) on \( \theta_1 - \theta_2 \). We 
will assume that \( F_p \) takes its minimum value (which we normalize to be 
0) when \( |\psi_1| = u_1 > 0, |\psi_2| = u_2 > 0 \) and 
\( \theta_1 - \theta_2 = 0 \). So, either there is no phase coupling 
(\( F_p \) is independent of \( \theta_1 - \theta_2 \)) and the choice of 
\( \theta_1 - \theta_2 = 0 \) is arbitrary, or the phase coupling is such as 
to encourage phase locking. (Note that the case of phase anti-locked fields 
can trivially be recovered from our analysis by mapping 
\( \psi_2 \mapsto -\psi_1 \)).

The field equations are obtained from \( F \) by demanding that the total 
free energy \( E = \int F dx_1 dx_2 \) is stationary with respect to all 
variations of \( \psi_1, \psi_2 \) and \( A_i \). A routine calculation yields 
\begin{gather}
    D_i D_i \psi_a = 2\pder{F_p}{\psi_a^{*}}
    \label{eq:4} \\
    \partial_i \left( \partial_i A_j - \partial_j A_i \right) = 
        e\sum\limits_{a=1}^{2}\Im\left( \psi_a^* D_j \psi_a \right)
    \label{eq:5}
\end{gather}
This triple of coupled nonlinear partial differential equations supports 
solutions of the form
\begin{gather}
    \psi_a = f_a(r)e^{i\theta} \nonumber \\
    (A_1, A_2) = \frac{a(r)}{r}(-\sin\theta, \cos\theta)
    label{eq:6}
\end{gather}
where \( f_1, f_2, a \) are real profile functions. Note that in some cases 
mixed gradient terms favour non-axially symmetric solutions. In this section 
we consider only axially symmetric vortices. Fields within the above ansatz 
satisfy the field equations if and only if the profile functions 
\( f_1(r), f_2(r), a(r) \) satisfy the coupled ordinary differential equation 
system
\begin{gather}
    f''_a + \frac{1}{r} f'_a - \frac{1}{r^2}(1+ea)^2 f_a = 
        \left. \pder{F_p}{|\psi_a|} \right|_{(u_1, u_2, 0)}
    \label{eq:7} \\
    a'' - \frac{1}{r} a' - e(1+ea)(f_1^2+f_2^2) = 0
    \label{eq:8}
\end{gather}
The solution we require, the vortex, has boundary behaviour 
\( f_a(r) \rightarrow u_a \), \( a(r) \rightarrow -1/e \) as 
\( r \rightarrow \infty \). So, for large \( r \), the quantities 
\begin{equation}
    \epsilon_a(r) = f_a(r) - u_a, \quad
    \alpha(r) = a(r) + \frac{1}{e}
    \label{eq:9}
\end{equation}
are small and so should, to leading order, satisfy the linearization of 
\eqref{eq:7},\eqref{eq:8} about \( (u_1, u_2, -1/e) \). That is, at large 
\( r \),
\begin{gather}
    \epsilon''_a + \frac{1}{r} \epsilon'_a = \sum\limits_{b=1}^{2}
        \mathcal{H}_{ab} \epsilon_b
    \label{eq:10} \\
    \alpha'' - \frac{1}{r} \alpha' - e^2(u_1^2 + u_2^2 )\alpha = 0
    \label{eq:11}
\end{gather}
where \( \mathcal{H} \) is the Hessian matrix of 
\( F_p(|\psi_1|, |\psi_2|, 0) \) about its minimum 
\begin{equation}
    \mathcal{H}_{ab} = \left. \pcder{F_p}{|\psi_a|}{|\psi_b|} 
        \right|_{(u_1, u_2, 0)}
    \label{eq:12}
\end{equation}
So \( \alpha \) decouples from \( \epsilon_1, \epsilon_2\) asymptotically, and 
we see immediately that
\begin{equation}
    \alpha(r) = q_0 r K_1(\mu_A r), \quad
    \mu_a = e\sqrt{u_1^2 + u_2^2}
    \label{eq:13}
\end{equation}
where \( K_n \) denotes the nth modified Bessel’s function of the second 
kind\( ^{23} \), and \( q_0 \) is an unknown real constant. Hence, at large,
\begin{equation}
    \vec{A} \sim \left( -\frac{1}{er} + q_0 K_1(\mu_A r) \right)
        (-\sin\theta, \cos\theta)
    \label{eq:14}
\end{equation}
Since, for all \( n \), 
\begin{equation}
    K_n(s) \sim \sqrt{\frac{\pi}{2s}}e^{-s} \text{ as } s \rightarrow \infty
    \label{eq:15}
\end{equation}
it follows that the magnetic field decays exponentially as a function of 
\( r \), with length scale (penetration depth)
\begin{equation}
    \lambda \equiv \frac{1}{\mu_A} = \frac{1}{e\sqrt{u_1^2 + u_2^2}}
    \label{eq:16}
\end{equation}

By contrast, \eqref{eq:10} represents, in general, a coupled pair of ordinary 
differential equations for \( \epsilon_1, \epsilon_2 \). Since 
\( (u_1, u_2, 0 ) \) is a \textit{minimum} of 
\( F_p(|\psi_1|, |\psi_2|, \theta_1 - \theta_2) \), the Hessian matrix is a 
\textit{positive definite} symmetric \( 2\times2 \) real matrix. Hence its 
eigenvalues, \( \mu_1^2, \mu_2^2\) say, are real and positive, and its 
eigenvectors, \( v_1, v_2 \) say, form an orthonormal basis for 
\( \mathbb{R} \). Expanding in the basis \( v_1, v_2 \)
\begin{equation}
    \epsilon(r) = \chi_1(r) v_1 + \chi_2(r) v_2
    \label{eq:17}
\end{equation}
we see that \( \chi_1, \chi_2 \) satisfy the uncoupled pair of ordinary 
differential equations
\begin{equation}
    \chi''_a + \frac{1}{r}\chi'_a = \mu_a^2 \chi_a
    \label{eq:18}
\end{equation}
whence 
\begin{equation}
    \chi_a(r) = q_a K_0(\mu_a r)
    \label{eq:19}
\end{equation}
for some (unknown) constants \( q_1, q_2 \). Since \( v_1, v_2 \) are 
orthonormal, there is an angle \( \Theta \), which we call the \emph{mixing 
angle}, such that the eigenvectors of \( \mathcal{H} \) are
\begin{equation}
    v_1 = \left( \begin{array}{c}
        \cos\Theta \\
        \sin\Theta
    \end{array} \right), \quad
    v_2 = \left( \begin{array}{c}
        -\sin\Theta \\
        \cos\Theta
    \end{array} \right)
    \label{eq:20}
\end{equation}
Hence, at large \( r \) the density fields behave as
\begin{gather}
    \psi_1 \sim \left[ u_1 + q_1\cos\Theta K_0(\mu_1 r) - 
        q_2\sin\Theta K_0(\mu_2 r) \right]e^{i\theta} \nonumber \\
    \psi_2 \sim \left[ u_2 + q_1\sin\Theta K_0(\mu_1 r) - 
        q_2\cos\Theta K_0(\mu_2 r) \right]e^{i\theta}
    \label{eq:21}
\end{gather}
where, once again, \( K_0 \) is a Bessel function.

From this analysis it follows that:
\begin{enumerate}
    \item In general there are three fundamental length scales in the problem 
        (in contrast to the two length scales of one-component Ginzburg-Landau 
        theory) which manifest themselves in the vortex asymptotics, namely 
        \( 1/\mu_A, 1/\mu_1 \) and \( 1/\mu_2 \).
    \item These are constructed from the vacuum expectation values \( u_a \) 
        of \( |\psi_a| \) (in the case of \( 1/\mu_A \)) and from the 
        eigenvalues of \( \mathcal{H} \), the Hessian matrix of \( F_p \) 
        about the vacuum (i.e. the ground state).
    \item \( 1/\mu_{A} \) can be interpreted as the London penetration length 
        of the magnetic field.
    \item However, unless the mixing angle \( \Theta \) is a multiple of 
        \( \pi/2, 1/\mu_1 \) and \( 1/\mu_2 \) cannot be interpreted as the 
        coherence lengths of \( \psi_1, \psi_2 \) in the usual sense. This is 
        because the normal modes of the field theory close to the vacuum are 
        not \( |\psi_a| - u_a \), but rather
        \[ 
            \chi_1 = (|\psi_1| - u_1)\cos\Theta - (|\psi_2| - u_2)\sin\Theta 
        \]
        \[ 
            \chi_2 = (|\psi_1| - u_1)\sin\Theta - (|\psi_2| - u_2)\cos\Theta 
        \]
        obtained by rotating through the mixing angle \( \Theta \), which is 
        also determined by \( \mathcal{H} \). Therefore in general (e.g. in 
        the presence of intercomponent Josephson coupling) for a one-flux 
        quantum axially symmetric vortex, the recovery of both fields 
        \( \psi_a \) at \emph{very} long range will be according to the same 
        exponential law, set by the smaller of the masses \( \mu_1, \mu_2 \); 
        One should use the representation in terms of the fields 
        \( \chi_{1,2} \) to be handle properly the two length scales 
        associated with the density recovery.
    \item This analysis tells us only about the vortex structure at large 
        \( r \). It gives no direct information on the vortex core, which is 
        important to understand quantitatively the nature of the vortex 
        interactions at intermediate and short distances which will be 
        studied numerically in the section \ref{ch:5}.
\end{enumerate}

Since the gauge field mediates a repulsive force between vortices, while the 
condensate fields mediate an attractive force, it is clear that we can read 
off from the above analysis the condition under which the intervortex force is 
attractive at long range: we require that \( 1/\mu_A \) is \emph{not} the 
longest of the three length scales, or, more explicitly, that (at least) one 
of the eigenvalues of \( \mathcal{H} \) should be less that 
\( \mu_A^2 = e^2(u_1^2 + u_2^2) \). We can predict an explicit formula for 
the long range two-vortex interaction potential, using the point vortex 
formalism\cite{bib:19} (a brief review of the method in given in Appendix C). 
This rests on the observation that, far from its core, the fields of the 
vortex are identical to those of a hypothetical point particle in a linear 
theory with two Klein-Gordon fields (\( \chi_1 \) and \( \chi_2 \) above) of 
mass and a vector field (A) of mass \( \mu_A \). The point particle carries 
scalar monopole charges \( 2\pi q_1 \) and \( 2\pi q_2 \) and a magnetic 
dipole moment \( 2\pi q_0 \). Two such hypothetical particles held distance 
\( r \) apart would experience an interaction potential
\begin{equation}
    V(r) = 2\pi\left[ q_0^2 K_0(\mu_A r) - q_1^2 K_0(\mu_1 r) - 
        q_2^2 K_0(\mu_2 r) \right]
    \label{eq:22}
\end{equation}

This formula reproduces the prediction explained above: the long range 
interaction will be attractive if (at least) one of \( \mu_1, \mu_2 \) is less 
than \( \mu_A \). 

One can ask, retrospectively, whether the approximation of \emph{linearizing} 
in the small quantities \( \alpha(r), \chi_1(r), \chi_2(r) \) is well 
justified. Rigorous analysis of the single component model\cite{bib:20} shows 
that if either of the scalar mode masses, say, exceeds \( 2\mu_A \), then 
quadratic terms in \( \alpha \) become comparable at large \( r \) with linear 
terms in \( \chi_2 \), so that the equation for \( \chi_2 \) should include 
extra terms. In this case, \( \chi_2 \) decays like \( K_0(\mu_A r)^2 \) 
rather than \( K_0(\mu_2 r) \). One should note, however that, unless
\( \mu_1 > 2\mu_A \) also, the leading term in \eqref{eq:21}, decaying like 
\( K_0(\mu_1 r) \), is still correct, and it is only the leading term which 
determines the nature (attractive or repulsive) of the intervortex 
interactions at long range. The case of interest to us is when the long-range 
force is attractive, that is, when at least one of \( \mu_1, \mu_2 \) is 
\emph{less than} \( \mu_A \), so the linearized analysis presented above 
suffices for our purposes.

\section{Симметричная модель \texorpdfstring{$ U(1) \times U(1) $}
  {U(1) x U(1)}}
\label{sec:3-1}

\section{Джозефоновский контакт}
\label{sec:3-2}

\subsection{Сравнение со случаем пассивной второй band (?)}
\label{subsec:3-2-1}

\section{Контакт плотность-плотность (?)}
\label{sec:3-3}

\section{Условия смешанного градиента}
\label{sec:3-4}
