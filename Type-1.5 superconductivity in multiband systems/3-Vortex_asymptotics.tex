\chapter{Вихревая асимптотика}
\label{ch:3}

Ключом к пониманию взаимодействия хорошо разделенных вихрей является анализ  
больших \( r \) асимптотического вихревого решения. Мы проанализируем эту 
проблему в контексте общей двухкомпонентной модели ГЛ, чья свободная энергия 
принимает форму
\begin{equation}
    F = \frac{1}{2}\left( D_i \psi_1 \right)^{*} D_i \psi_1 + 
        \frac{1}{2}\left( D_i \psi_2 \right)^{*} D_i \psi_2 + 
        \frac{1}{2}\left( \partial_1 A_2 - \partial_2 A_1 \right)^2 + F_p
    \label{eq:3}
\end{equation}
где \( F_p \) содержит все не-градиентные члены (в частности, но не 
ограничиваясь, джозефсоновским и плотность-плотность членами взаимодействия). 
Эта свободная энергия соответствует \eqref{eq:2} в случае \( \nu = 0 \). Мы 
покажем в разделе IIID как работать со смешанными градиентными членами. Точная 
форма \( F_p \) в данном разделе не является решающим фактором для анализа.
При калибровочной инвариантности, это может зависеть от конденсатов только 
через \( |\psi_1|, |\psi_2| \) и (если конденсаты не являются независимо 
сохраняющимися) на \( \theta_1 - \theta_2 \). Будем считать, что \( F_p \) 
принимает минимальное значение (которое должно быть приведено к 0), когда 
\( |\psi_1| = u_1 > 0, |\psi_2| = u_2 > 0 \) и \( \theta_1 - \theta_2 = 0 \).
Таким образом, либо нет фазовой связи (\( F_p \) не зависит от 
\( \theta_1 - \theta_2 \)) и выбор \( \theta_1 - \theta_2 = 0 \) произволен, 
или фазовая связь является такой, чтобы стимулировать фазовую синхронизацию.
(Отметим, что случай фазовых рассинхронизаций полей могут быть тривиальным 
образом восстановлены из нашего анализа путём отображения 
\( \psi_2 \mapsto -\psi_1 \)).

Уравнения поля получаются из \( F \), требуя, чтобы общая свободная энергия 
\( E = \int F dx_1 dx_2 \) являлась стационарной по отношению ко всем 
изменениям \( \psi_1, \psi_2 \) и \( A_i \). Обычный расчёт даёт 
\begin{gather}
    D_i D_i \psi_a = 2\pder{F_p}{\psi_a^{*}}
    \label{eq:4} \\
    \partial_i \left( \partial_i A_j - \partial_j A_i \right) = 
        e\sum\limits_{a=1}^{2}\Im\left( \psi_a^* D_j \psi_a \right)
    \label{eq:5}
\end{gather}
Решение этой тройки связанных нелинейных дифференциальных уравнений в частных 
производных можно представить в виде
\begin{gather}
    \psi_a = f_a(r)e^{i\theta} \nonumber \\
    (A_1, A_2) = \frac{a(r)}{r}(-\sin\theta, \cos\theta)
    label{eq:6}
\end{gather}
где \( f_1, f_2, a \) вещественные функции профиля. Примем во внимание, что 
в некоторых случаях смешанные градиентные слагаемые выступают за не 
осесимметричное решение. В этом разделе мы рассмотрим только 
аксиально-симметричные вихри. Поля в пределах указанного выше подхода 
удовлетворяют уравнениям поля, если и только если функция профиля 
\( f_1(r), f_2(r), a(r) \) удовлетворяют взаимной системе дифференциальных 
уравнений
\begin{gather}
    f''_a + \frac{1}{r} f'_a - \frac{1}{r^2}(1+ea)^2 f_a = 
        \left. \pder{F_p}{|\psi_a|} \right|_{(u_1, u_2, 0)}
    \label{eq:7} \\
    a'' - \frac{1}{r} a' - e(1+ea)(f_1^2+f_2^2) = 0
    \label{eq:8}
\end{gather}
Потребуем чтобы вихревое решение имело поведение на границе
\( f_a(r) \rightarrow u_a \), \( a(r) \rightarrow -1/e \) при
\( r \rightarrow \infty \). Так, для больших \( r \) величины
\begin{equation}
    \epsilon_a(r) = f_a(r) - u_a, \quad
    \alpha(r) = a(r) + \frac{1}{e}
    \label{eq:9}
\end{equation}
мали и так должны быть в старшем порядке, удовлетворяющие линеаризации 
\eqref{eq:7},\eqref{eq:8} относительно \( (u_1, u_2, -1/e) \). То есть, при 
больших \( r \),
\begin{gather}
    \epsilon''_a + \frac{1}{r} \epsilon'_a = \sum\limits_{b=1}^{2}
        \mathcal{H}_{ab} \epsilon_b
    \label{eq:10} \\
    \alpha'' - \frac{1}{r} \alpha' - e^2(u_1^2 + u_2^2 )\alpha = 0
    \label{eq:11}
\end{gather}
где \( \mathcal{H} \) является матрицей Гессе \( F_p(|\psi_1|, |\psi_2|, 0) \) 
и его минимум
\begin{equation}
    \mathcal{H}_{ab} = \left. \pcder{F_p}{|\psi_a|}{|\psi_b|} 
        \right|_{(u_1, u_2, 0)}
    \label{eq:12}
\end{equation}
Так \( \alpha \) асимптотически отделяется от \( \epsilon_1, \epsilon_2\) и 
сразу видно, что
\begin{equation}
    \alpha(r) = q_0 r K_1(\mu_A r), \quad
    \mu_a = e\sqrt{u_1^2 + u_2^2}
    \label{eq:13}
\end{equation}
где \( K_n \) обозначает n-ую модифицированную функцию Бесселя второго порядка 
\cite{bib:23}, и \( q_0 \) неизвестная действительная постоянная. Таким образом
\begin{equation}
    \vec{A} \sim \left( -\frac{1}{er} + q_0 K_1(\mu_A r) \right)
        (-\sin\theta, \cos\theta)
    \label{eq:14}
\end{equation}
Так, для всех \( n \), 
\begin{equation}
    K_n(s) \sim \sqrt{\frac{\pi}{2s}}e^{-s} \text{ as } s \rightarrow \infty
    \label{eq:15}
\end{equation}
отсюда вытекает, что магнитное поле затухает экспоненциально в зависимости от 
\( r \), с масштабной величиной (глубина проникновения)
\begin{equation}
    \lambda \equiv \frac{1}{\mu_A} = \frac{1}{e\sqrt{u_1^2 + u_2^2}}
    \label{eq:16}
\end{equation}

С другой стороны, \eqref{eq:10} представляет, в общем, пару связанных 
обыкновенных дифференциальных уравнений для \( \epsilon_1, \epsilon_2 \). Так 
как \( (u_1, u_2, 0 ) \) является \textit{минимумом} от 
\( F_p(|\psi_1|, |\psi_2|, \theta_1 - \theta_2) \), где матрица Гессе является  
симметричной и \textit{положительно определенной} действительной матрицей 
размера \( 2\times2 \). Следовательно, её собственные числа 
\( \mu_1^2, \mu_2^2\), допустим вещественны и положительны, и её собственные 
векторы \( v_1, v_2 \) формируют ортонормированный базис на \( \mathbb{R} \). 
Расширяя базис \( v_1, v_2 \)
\begin{equation}
    \epsilon(r) = \chi_1(r) v_1 + \chi_2(r) v_2
    \label{eq:17}
\end{equation}
мы видим, что \( \chi_1, \chi_2 \) удовлетворяет несвязанной паре обычных 
дифференциальных уравнений
\begin{equation}
    \chi''_a + \frac{1}{r}\chi'_a = \mu_a^2 \chi_a
    \label{eq:18}
\end{equation}
откуда
\begin{equation}
    \chi_a(r) = q_a K_0(\mu_a r)
    \label{eq:19}
\end{equation}
где \( q_1, q_2 \) некоторые (неизвестные) константы. Т.к. \( v_1, v_2 \) 
ортонормированы, существует угол \( \Theta \), называемый 
\emph{углом смешивания}, такой, что собственные векторы \( \mathcal{H} \) 
являются
\begin{equation}
    v_1 = \left( \begin{array}{c}
        \cos\Theta \\
        \sin\Theta
    \end{array} \right), \quad
    v_2 = \left( \begin{array}{c}
        -\sin\Theta \\
        \cos\Theta
    \end{array} \right)
    \label{eq:20}
\end{equation}
Так, при больших \( r \) поля плотности ведут себя как 
\begin{gather}
    \psi_1 \sim \left[ u_1 + q_1\cos\Theta K_0(\mu_1 r) - 
        q_2\sin\Theta K_0(\mu_2 r) \right]e^{i\theta} \nonumber \\
    \psi_2 \sim \left[ u_2 + q_1\sin\Theta K_0(\mu_1 r) - 
        q_2\cos\Theta K_0(\mu_2 r) \right]e^{i\theta}
    \label{eq:21}
\end{gather}
где, ещё раз, \( K_0 \) -- функция(и) Бесселя.

Из этого анализа следует, что:
\begin{enumerate}
    \item В целом есть три фундаментальные масштабные длины в задаче (в 
        отличие от двух масштабных длин в однокомпонентной теории 
        Гинзбурга-Ландау), которые проявляются в вихревых асимптотиках, а 
        именно \( 1/\mu_A, 1/\mu_1 \) и \( 1/\mu_2 \).
    \item Они строятся из vacuum expectation values \( u_a \) 
        jn \( |\psi_a| \) (в случае с \( 1/\mu_A \)) и из собственных значений 
        \( \mathcal{H} \), матрица Гессе \( F_p \) about the vacuum (т.е. 
        основного состояния).
    \item \( 1/\mu_{A} \) может быть интерпретирована как лондоновская глубина 
        проникновения магнитного поля.
    \item Однако если угол смешивания \( \Theta \) не является кратным  
        \( \pi/2, 1/\mu_1 \) и \( 1/\mu_2 \) не может быть истолкована как 
        длина когерентности \( \psi_1, \psi_2 \) в обычном смылсе. This is 
        because the normal modes of the field theory close to the vacuum are 
        not \( |\psi_a| - u_a \), but rather
        \[ 
            \chi_1 = (|\psi_1| - u_1)\cos\Theta - (|\psi_2| - u_2)\sin\Theta 
        \]
        \[ 
            \chi_2 = (|\psi_1| - u_1)\sin\Theta - (|\psi_2| - u_2)\cos\Theta 
        \]
        obtained by rotating through the mixing angle \( \Theta \), which is 
        also determined by \( \mathcal{H} \). Therefore in general (e.g. in 
        the presence of intercomponent Josephson coupling) for a one-flux 
        quantum axially symmetric vortex, the recovery of both fields 
        \( \psi_a \) at \emph{very} long range will be according to the same 
        exponential law, set by the smaller of the masses \( \mu_1, \mu_2 \); 
        One should use the representation in terms of the fields 
        \( \chi_{1,2} \) to be handle properly the two length scales 
        associated with the density recovery.
    \item Этот анализ говорит нам только о вихревой структуры при больших
        \( r \). Это не дает прямую информацию о ядре вихря, чтобы 
        количественно понять природу вихревых взаимодействий на переходных 
        и коротких расстояниях, которые будут численно изучены в разделе 
        \ref{ch:5}.
\end{enumerate}

Since the gauge field mediates a repulsive force between vortices, while the 
condensate fields mediate an attractive force, it is clear that we can read 
off from the above analysis the condition under which the intervortex force is 
attractive at long range: we require that \( 1/\mu_A \) is \emph{not} the 
longest of the three length scales, or, more explicitly, that (at least) one 
of the eigenvalues of \( \mathcal{H} \) should be less that 
\( \mu_A^2 = e^2(u_1^2 + u_2^2) \). We can predict an explicit formula for 
the long range two-vortex interaction potential, using the point vortex 
formalism\cite{bib:19} (a brief review of the method in given in Appendix C). 
This rests on the observation that, far from its core, the fields of the 
vortex are identical to those of a hypothetical point particle in a linear 
theory with two Klein-Gordon fields (\( \chi_1 \) and \( \chi_2 \) above) of 
mass and a vector field (A) of mass \( \mu_A \). The point particle carries 
scalar monopole charges \( 2\pi q_1 \) and \( 2\pi q_2 \) and a magnetic 
dipole moment \( 2\pi q_0 \). Two such hypothetical particles held distance 
\( r \) apart would experience an interaction potential
\begin{equation}
    V(r) = 2\pi\left[ q_0^2 K_0(\mu_A r) - q_1^2 K_0(\mu_1 r) - 
        q_2^2 K_0(\mu_2 r) \right]
    \label{eq:22}
\end{equation}

Эта формула воспроизводит предсказанное выше объяснение: взаимодействие на 
больших расстояниях будет притягивающим, если (по крайней мере) один из 
\( \mu_1, \mu_2 \) меньше чем \( \mu_A \). 

One can ask, retrospectively, whether the approximation of \emph{linearizing} 
in the small quantities \( \alpha(r), \chi_1(r), \chi_2(r) \) is well 
justified. Rigorous analysis of the single component model\cite{bib:20} shows 
that if either of the scalar mode masses, say, exceeds \( 2\mu_A \), then 
quadratic terms in \( \alpha \) become comparable at large \( r \) with linear 
terms in \( \chi_2 \), so that the equation for \( \chi_2 \) should include 
extra terms. In this case, \( \chi_2 \) decays like \( K_0(\mu_A r)^2 \) 
rather than \( K_0(\mu_2 r) \). One should note, however that, unless
\( \mu_1 > 2\mu_A \) also, the leading term in \eqref{eq:21}, decaying like 
\( K_0(\mu_1 r) \), is still correct, and it is only the leading term which 
determines the nature (attractive or repulsive) of the intervortex 
interactions at long range. The case of interest to us is when the long-range 
force is attractive, that is, when at least one of \( \mu_1, \mu_2 \) is 
\emph{less than} \( \mu_A \), so the linearized analysis presented above 
suffices for our purposes.

\section{Симметричная модель \texorpdfstring{$ U(1) \times U(1) $}
  {U(1) x U(1)}}
\label{sec:3-1}

\section{Джозефсоновский контакт}
\label{sec:3-2}

\subsection{Сравнение со случаем пассивной второй щелью (зоной)}
\label{subsec:3-2-1}

\section{Контакт типа плотность-плотность (?)}
\label{sec:3-3}

\section{Условия смешанного градиента}
\label{sec:3-4}
