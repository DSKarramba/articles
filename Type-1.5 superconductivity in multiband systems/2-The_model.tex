\chapter{Модель}
\label{ch:2}

\section{Функционал свободной энергии}
\label{sec:2-1}

Тип 1,5 изучается с помощью следующей двухкомпонентного функционала свободной 
энергии Гинзбурга-Ландау (ТГЛ):
\begin{align}
	F = \frac{1}{2}(D\psi_1)(D\psi_1)^* + \frac{1}{2}(D\psi_2)(D\psi_2)^* - 
		\nu Re\left( (D\psi_1)(D\psi_2)^* \right) + 
		\frac{1}{2}\left(\nabla\times\vec{A}\right)^2 + F_p
	\label{eq:1}
\end{align}

Здесь \( D = \nabla + ie\vec{A} \) и \( \psi_a = |\psi_a|e^{i\theta_a} \), 
\( a = 1,2 \), представляет собой две сверхтекучих компоненты, которые в 
двухщелевом сверхпроводнике соответствуют двум сверхтекучим плотностям в 
в различных диапазонах. Слагаемое \( F_p \) может содержать в нашем 
анализе произвольный набор не градиентных членов.

Особая форма двухкомпонентной модели ГЛ точно выведенная 
\cite{bib:8,bib:9,bib:10} для двухщелевых сверхпроводников является:
\begin{align}
	F = & \frac{1}{2}(D\psi_1)(D\psi_1)^* + \frac{1}{2}(D\psi_2)(D\psi_2)^* - 
		\nu Re\left\{ (D\psi_1)(D\psi_2)^* \right\} + 
		\frac{1}{2}\left(\nabla\times\vec{A}\right)^2 + \alpha_1|\psi_1|^2 + 
		\nonumber \\
		& + \frac{1}{2}\beta_1|\psi_1|^4 + \alpha_2|\psi_2|^2 + 
		\frac{1}{2}\beta_2|\psi_2|^4 + \eta_1|\psi_1||\psi_2|
		\cos(\theta_1-\theta_2) + \eta_2|\psi_1|^4|\psi_2|^2
	\label{eq:2}
\end{align}

Первые два слагаемых представляют стандартный градиентный член 
Гинзбурга-Ландау, второе слагаемое представляет смешанные градиентные 
взаимодействия, которые появляются в двухщелевых сверхпроводниках с примесным 
рассеянием \cite{bib:8,bib:9}. Следующее член является плотностью энергии 
магнитного поля, а остальные слагаемые представляют эффективный потенциал. 
Здесь же отметим, что \( \alpha_1 \) и \( \alpha_2 \) могут инвертировать знак 
при различных температурах. Режим, где \( \alpha_1 \) положительно, в то время 
\( \alpha_2 \) отрицательно, соответствует ситуации, когда один из группы 
не имеет собственной сверхпроводимости, но, тем не менее имеет некоторые 
сверхтекучие плотности из-за межзонного туннелирования Джозефсона, которая 
представлена \( \eta|\psi_1||\psi_2|\cos(\theta_1-\theta_2) \) слагаемым. 
Поведения типа 1,5 в этом режиме был исследован в\cite{bib:2}. В этой работе 
мы в основном сосредоточимся на ситуации когда обе зоны являются активными, 
т.е. \( \alpha_{1,2} < 0 \). Для общности добавим слагаемое более высокого 
порядка связи плотность-плотность \( \eta_2|\psi_1|^2|\psi_2|^2 \). Мы также 
рассмотрим случай независимо сохраняющихся конденсатов, где третий и девятый 
члены в \eqref{eq:2} запрещены на основании симметрии, то есть 
\( \nu = \eta_1 = 0 \) (см. также замечание\cite{bib:21}). Эквивалентность 
между нашими единицами и общепринятого даётся в Приложении \ref{ch:A}.

Точно выведенная модель ГЛ \eqref{eq:2} требует малость полей \( |\psi_a| \). 
Однако это не требует в принципе от \( \alpha_a \) менять знак при той же 
температуре. Кроме того, как и в случае однокомпонентной теории ГЛ, мы 
ожидаем, что модель \eqref{eq:2} даёт во многих случаях приемлемую картину 
в низкотемпературном режиме. Фактически, наш анализ может в некоторых случаях 
дать качественную картину для случая, когда одно из полей не обладает 
эффективным потенциалом ГЛ-типа, так как режим, где одна из зон в Лондоновском 
приближении (т.е. она не обладает эффективным потенциалом ГЛ, но небольшое ядро 
вихря моделируется резкой границей отсечки) может быть восстановлена из 
нашего анализа, как предельный случай. Как будет ясно ниже из анализа этого 
режима он также поддерживает сверхпроводимость типа 1,5.

\begin{table}[ht]
    \centering
    \begin{tabular}{|C{.18}|C{.22}|C{.22}|C{.35}|}
        \hline
        & однокомпонентный тип I & однокомпонентный тип II & 
        многокомпонентный тип 1,5 \\ \hline

        Масштабная характерная длина & Глубина проникновения \( \lambda \) и 
            длина когерентности \( \xi \) 
            (\( \frac{\lambda}{\xi} < \frac{1}{\sqrt{2}} \)) &
        Глубина проникновения \( \lambda \) и 
            длина когерентности \( \xi \) 
            (\( \frac{\lambda}{\xi} > \frac{1}{\sqrt{2}} \)) &
        Две характеристические вариации плостностей масштабных величин 
            \( \xi_1, \xi_2 \) и глубина проникновения \( \lambda \), 
            немонотонное взаимодействие вихря происходит в этих системах, как 
            правило, когда \( \xi_1 < \sqrt{2}\lambda < \xi_2 \) \\ \hline

        Межвихревое взаимодействие & Притягивающее & Отталкивающее & 
            Притягивающее на больших расстояниях и отталкивающее на коротких 
        \\ \hline

        Энергия сверхпроводников/граница фаз & Положительная & Отрицательная &
            При довольно общих условиях отрицательной энергии сверхпроводника / 
            границы раздела внутри вихревого скопления, но положительна вне (!!).

        \\ \hline

        Магнитное поле, необходимое для образования вихря & Больше чем 
            критическое термодинамическое магнитное поле & Меньше чем 
            критическое термодинамическое магнитное поле &
        В различных случаях либо (i) меньше, чем критическое термодинамическое 
            магнитное поле или (ii) больше, чем критическое термодинамическое
            магнитное поле для одиночного вихря, но меньше критического 
            магнитного поля для вихревого образования некоторого критического 
            размера.

        \\ \hline

        Этапы внешнего магнитного поля & (i) Эффект Мейснера для слабых полей; 
            (ii) Макроскопически большее нормальное состояние в больших полях. 
            Фазовый переход первого рода между сверхпроводящим (Мейснер) и 
            нормальным состоянием & (i) Эффект Мейснера для слабых полей; (ii) 
            вихревые решётки/жидкости при больших полях. Фазовые переходы 
            второго рода между состоянием Мейснера и вихревым, а также 
            между вихревым и нормальным. & (i) Эффект Мейснера для слабых 
            полей. (ii) "полумейсснеровское состояние": вихревые образования 
            сосуществуют с областью Мейснера на промежуточных полях. (iii) 
            Вихревые решётки/жидкости при больших полях. Вихревая форма путём 
            фазового перехода первого рода. Переход от вихревого состояния в 
            нормальное состояния является переходом второго рода (!!).
        \\ \hline

        Энергия \( E(N) \) или N-квантовые аксиально-симметричные вихревые 
            решения(!) & \( \frac{E(N)}{N} < \frac{E(N-1)}{N-1} \) для всех 
            \( N \). Вихри сливаются в единый \( N \)-квантовой мегавихрь &
        \( \frac{E(N)}{N} > \frac{E(N-1)}{N-1} \) для всех \( N \). 
            \( N \)-квантовый вихрь распадается на \( N \) бесконечно 
            разделенных одноквантовых вихрей & Здесь характеристическое число 
            \( N \) такое, что \( \frac{E(N)}{N} < \frac{E(N-1)}{N-1} \) для 
            \( N < N_c \), тогда как 
            \( \frac{E(N)}{N} > \frac{E(N-1)}{N-1} \) для \( N > N_c \). 
            N-квантовые вихри распадаются на вихревые образования.
        \\ \hline

    \end{tabular}
    \caption{Основные характеристики чистых сверхпроводников первого, второго и 
        1,5 рода. Здесь указаны наиболее распространённые единицы измерения 
        используемые в теории ГЛ, которые подразделяются на первый и второй 
        род в однокомпонентной теории параметром \( \kappa_c = 1/\sqrt{2} \)}
\end{table}

\section{Основные свойства вихревых возбуждений}
\label{sec:2-2}

Единственные вихревые решения модели \eqref{eq:2}, которые имеют конечную 
энергию на единицу длины являются целыми N-поточными квантовыми вихрями, 
которые имеют следующие фазовые обороты вдоль контура \( l \) вокруг вихря: 
\( \oint\limits_l \nabla\theta_l = 2\pi N \), 
\( \oint\limits_l \nabla\theta_l = 2\pi N \). Вихрь с разным фазовой закруткой 
несёт дробное число, кратное кванту магнитного потока и обладающий
расходящейся энергией с размером системы. Решения подробно исследованы в
\cite{bib:22}.

В дальнейшем мы исследуем только целочисленные вихревые решения, которые 
являются энергетически выгодными объектами производимыми с помощью внешнего 
поля в объёмном сверхпроводнике.