\chapter{Модель}
\label{ch:2}

\section{Функционал свободной энергии}
\label{sec:2-1}

Тип 1.5 изучается с помощью следующей двухкомпонентного функционала 
свободной энергии Гинзбурга-Ландау (ТГЛ):
\begin{align}
	F = \frac{1}{2}(D\psi_1)(D\psi_1)^* + \frac{1}{2}(D\psi_2)(D\psi_2)^* - 
		\nu Re\left( (D\psi_1)(D\psi_2)^* \right) + 
		\frac{1}{2}\left(\nabla\times\vec{A}\right)^2 + F_p
	\label{eq:1}
\end{align}

Здесь \( D = \nabla + ie\vec{A} \) и \( \psi_a = |\psi_a|e^{i\theta_a} \), 
\( a = 1,2 \), представляет собой две сверхтекучих компоненты, которые в 
двухзонном сверхпроводнике соответствуют двум сверхтекучим densities в 
в различных диапазонах. Слагаемое \( F_p \) может содержать в нашем 
анализе произвольный набор не градиентных членов.

Особая форма двухкомпонентной модели ГЛ точно 
выведенная\cite{bib:8,bib:9,bib:10} для двухзонных сверхпроводников является:
\begin{align}
	F = & \frac{1}{2}(D\psi_1)(D\psi_1)^* + \frac{1}{2}(D\psi_2)(D\psi_2)^* - 
		\nu Re\left\{ (D\psi_1)(D\psi_2)^* \right\} + 
		\frac{1}{2}\left(\nabla\times\vec{A}\right)^2 + \alpha_1|\psi_1|^2 + 
		\nonumber \\
		& + \frac{1}{2}\beta_1|\psi_1|^4 + \alpha_2|\psi_2|^2 + 
		\frac{1}{2}\beta_2|\psi_2|^4 + \eta_1|\psi_1||\psi_2|
		\cos(\theta_1-\theta_2) + \eta_2|\psi_1|^4|\psi_2|^2
	\label{eq:2}
\end{align}

Первые два слагаемых представляют стандартный градиентный член 
Гинзбурга-Ландау, второе слагаемое представляет смешанные градиентные 
взаимодействия, которые появляются в двухзонных сверхпроводниках с примесным 
рассеянием \cite{bib:8,bib:9}. Следующее член является плотностью энергии 
магнитного поля, а остальные слагаемые представляют эффективный потенциал. 
Здесь же отметим, что \( \alpha_1 \) и \( \alpha_2 \) могут инвертировать знак 
при различных температурах. Режим где \( \alpha_1 \) положительно, в то время 
\( \alpha_2 \) отрицательно, что соответствует ситуации, когда один из группы 
не имеет собственной сверхпроводимости, но, тем не менее имеет некоторые 
сверхтекучие density из-за межзонного туннелирования Джосефсона, которая 
представлена \( \eta|\psi_1||\psi_2|\cos(\theta_1-\theta_2) \) слагаемым. 
Поведения типа 1.5 в этом режиме был исследован в\cite{bib:2}. В этой работе 
мы в основном сосредоточимся на ситуации когда обе зоны являются активными, 
т.е. \( \alpha_{1,2} < 0 \). Для общности добавим слагаемое более высокого 
порядка density-density coupling \( \eta_2|\psi_1|^2|\psi_2|^2 \). Мы также 
рассмотрим случай независимо сохраняющихся condensates, где третий и девятый 
члены в \eqref{eq:2} запрещены на основании симметрии, то есть 
\( \nu = \eta_1 = 0 \) (см. также замечение\cite{bib:21}). Эквивалентность 
между нашими единицами и общепринятого даётся в Приложении \ref{ch:A}.

Точно выведенная модель ГЛ \eqref{eq:2} необходимо малость полей \( |\psi_a| \). 
Однако это не требует в принципе от \( \alpha_a \) менять знак при той же 
температуре. Кроме того, как и в случае однокомпонентной теории ГЛ, мы 
ожидаем, что модель \eqref{eq:2} даёт во многих случаях приемлемую картину в 
низкотемпературном режиме. Фактически, наш анализ может в некоторых случаях 
дать качественную картину для случая, когда одно из полей не обладает 
эффективным потенциалом ГЛ-типа, так как режим, где одна из зон в Лондоновском 
приближении (т.е. она не обладает эффективным потенциалом ГЛ, но небольшое ядро 
вихря моделируется резкой границой отсечки (!!)) может быть восстановлена из нашего 
анализа, как предельный случай. Как будет ясно ниже из анализа этого режима он 
также поддерживает сверхпроводимость типа 1.5.

-- ТАБЛИЦА --

\section{Основные свойства вихревых возбуждений}
\label{sec:2-2}

Единственные вихревые решения модели \eqref{eq:2}, которые имеют конечную 
энергию на единицу длины являются целыми N-поточными квантовыми вихрями, 
которые имеют следующие phase windings вдоль контура \( l \) вокруг вихря: 
\( \oint\limits_l \nabla\theta_l = 2\pi N \), 
\( \oint\limits_l \nabla\theta_l = 2\pi N \). Vortices with differing phase 
windings carry a fractional multiple of the magnetic flux quantum and have 
energy divergent with the system size. Решения подробно исследованы в
\cite{bib:22}.

В дальнейшем мы исследуем только integer flux vortex решений, которые 
являются энергетически выгодными objects to produce by means of внешнего поля 
в объёмном сверхпроводнике.