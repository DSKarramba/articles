\chapter{Модель}
\label{ch:2}

\section{Функционал свободной энергии}
\label{sec:2-1}

Тип 1.5 изучается с помощью следующей двухкомпонентного функционала 
свободной энергии Гинзбурга-Ландау (ТГЛ):
\begin{align}
	F = \frac{1}{2}(D\psi_1)(D\psi_1)* + \frac{1}{2}(D\psi_2)(D\psi_2)* - 
		\nu Re\left( (D\psi_1)(D\psi_2)* \right) + 
		\frac{1}{2}\left(\nabla\times\vec{A}\right)^2 + F_p
	\label{eq:1}
\end{align}

Здесь \( D = \nabla + ie\vec{A} \) и \( \psi_a = |\psi_a|e^{i\theta_a} \), 
\( a = 1,2 \), представляет собой две сверхтекучих компоненты, которые в 
двухзонном сверхпроводнике соответствуют двум сверхтекучим densities в 
в различных диапазонах. Слагаемое \( F_p \) может содержать в нашем 
анализе произвольный набор не градиентных членов.

Особая форма двухкомпонентной модели ГЛ точно выведенная\cite{bib:8,bib:9,bib:10} для 
двухзонных сверхпроводников является:
\begin{gather}
	F = \frac{1}{2}(D\psi_1)(D\psi_1)* + \frac{1}{2}(D\psi_2)(D\psi_2)* - 
		\nu Re\left( (D\psi_1)(D\psi_2)* \right) + 
		\frac{1}{2}\left(\nabla\times\vec{A}\right)^2 + \alpha_1|\psi_1|^2 + 
		\frac{1}{2}\beta_1|\psi_1|^4 + \alpha_2|\psi_2|^2 + 
		\frac{1}{2}\beta_2|\psi_2|^4 + \eta_1|\psi_1||\psi_2|
		\cos(\theta_1-\theta_2) + \eta_2|\psi_1|^4|\psi_2|^2
	\label{eq:2}
\end{gather}

Первые два слагаемых представляют стандартный градиентный член Гинзбурга-Ландау, 
второе слагаемое представляет смешанные градиентные взаимодействия, которые 
появляются в двухзонных сверхпроводниках с примесным рассеянием
\cite{bib:8,bib:9}. Следующее слагаемое 

\section{Основные свойства вихревых возбуждений}
\label{sec:2-2}
