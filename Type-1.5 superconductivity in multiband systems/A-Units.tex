\chapter{Единицы измерения}
\label{ch:A}

В этом разделе мы дадим явное отображение нашего представления модели
ГЛ к более общему виду представленного в учебниках. Рассмотрим модель
Гинзбурга-Ландау в общем представлении
\begin{align}
	F = & \frac{\hbar^2}{2m_1} \abs{\left( \nabla - i\frac{e^*}{\hbar c} A\right)
	  \psi_1}^2 + \frac{\hbar^2}{2m_2} \abs{\left( \nabla - i\frac{e^*}{\hbar c}A
	  \right) \psi_2}^2 - \nu\hbar^2 \Re\left\{ \left( \nabla - i\frac{e^*}{\hbar c}
	  \right)\psi_1  \cdot \left( \nabla + i\frac{e^*}{\hbar c}A \right)
	  \psi^*_2\right\} + \nonumber \\
  & + \frac{1}{8\pi}\left( \nabla\times A \right)^2 + \alpha_1\abs{\psi_1}^2 +
		\frac{1}{2}\beta_1\abs{\psi_1}^4 + \alpha_2\abs{\psi_2}^2 + \frac{1}{2}
		\beta_2\abs{\psi_2}^4 - \nonumber \\
	& - \eta_1\abs{\psi_1}\abs{\psi_2} \cos(\theta_1 - \theta_2) +
		\eta_2\abs{\psi_1}^2\abs{\psi_2}^2. \label{eq:A-1}
\end{align}

Давайте определим обозначенные величины
\begin{gather}
	\tilde{F} = \frac{4\pi}{\hbar^2 c^2}F \nonumber \\
	\tilde{A} = -\frac{A}{\hbar c} \nonumber \\
	\tilde{\psi}_a = \sqrt{\frac{4\pi}{m_a c^2}}\psi_a \nonumber \\
	\tilde{\nu} = \frac{m_1m_2}\nu \nonumber \\
	\tilde{\alpha}_a = \frac{m_a}{\hbar^2}\alpha_a \nonumber \\
	\tilde{\beta}_a = \frac{m^2_a c^2}{4\pi\hbar^2}\beta_a \nonumber \\
	\tilde{\eta}_1 = \frac{\sqrt{m_1 m_2}}{\hbar^2}\eta_1 \nonumber \\
	\tilde{\eta}_2 = \frac{m_1 m_2 c^2}{4\pi\hbar^2}\eta_2. \label{eq:A-2}
\end{gather}

Тогда
\begin{align}
	\tilde{F} = & \frac{1}{2}\abs{\left( \nabla + ie^*\tilde{A} \right)
	  \tilde{\psi}_2}^2 + \frac{1}{2}\abs{\left( \nabla + ie^*\tilde{A} \right)
	  \tilde{\psi}_2}^2 - \tilde{\nu} \Re\left\{ \left( \nabla + ie^*\tilde{A}
	  \right)\tilde{\psi}_1 \cdot \left( \nabla + ie^*\tilde{A} \right)
	  \tilde{\psi}^*_2 \right\} + \nonumber \\
	& \frac{1}{2}\abs{\nabla\times\tilde{A}}^2 - \tilde{\alpha}_1
	  \abs{\tilde{\psi}_1}^2 + \frac{\tilde{\beta}_1}{2}\abs{\tilde{\psi}_1}^2 +
		\tilde{\alpha}_2\abs{\tilde{\psi}_2}^2 + \nonumber \\
  & \frac{\tilde{\beta}_2}{2}\abs{\tilde{\psi}_2}^2 + \tilde{\eta}_1
		\abs{\tilde{\psi}_1}\abs{\tilde{\psi}_2}\cos(\theta_1 - \theta_2) +
		\tilde{\eta}_2\abs{\tilde{\psi}_1}^2 \abs{\tilde{\psi}_2}^2, \label{eq:A-3}
\end{align}
которая, опустив знак тильды, совпадает с представленным \eqref{eq:2} в
данной статье. На протяжении всей статьи, предполагается, что группа(!) 1
активна, то есть, \( \alpha_1 < 0 \). Перепишем уравнение \eqref{eq:2} для
\( F \), так, чтобы \( \alpha_1 \) была нормирована на \( -1 \) и
\( \beta_1 \) нормирована на \( 1 \). Это может быть достигнуто следующим
образом. Напомним (см. раздел \ref{sec:3-1}), что в отсутствии межзонной связи (т.е.
когда \( \eta_1 = \eta_2 = \nu = 0 \)) condensate 1 имеет затухание линейного
масштаба \( 1/\hat{\mu}_1 = (-4\alpha_1)^{-1/2} \). Эта scale обычно
определяется длиной когерентности
\begin{equation}
	\hat{\xi}_1 = \frac{\sqrt{2}}{\hat{\mu}_1} = \frac{1}{\sqrt{-2\alpha_1}}
	\label{eq:A-4}
\end{equation}

Ещё раз подчеркнём, что в присутствии межзонной связи, \( \hat{\xi}_1 \) вне
длины когерентности condensate 1. Таково определение параметра с шляпой,
напоминая, что это оригинальная длина когерентности только в несвязанном
случае. Напомним также, что vacuum density of condensate в несвязанной модели
\begin{equation}
	\hat{u}_1 = \sqrt{\frac{-\alpha_1}{\beta_1}}
	\label{eq:A-5}
\end{equation}

Наша второе перемасштабирование \( \sqrt{2}\hat{\xi}_1 \) как единицы длины
и \( \hat{u}_1 \) как единицы condensate density (вместе с компенсирующим
масштабированием \( F \), \( e^* \) и \( A \)). Конкретно говоря
\begin{gather}
	\bar{r} = \frac{r}{\sqrt{2}\hat{\xi}_1} = \sqrt{-\alpha_1 r} \nonumber \\
	\bar{F} = \frac{2\hat{\xi}^2_1}{\hat{u}^4_1}F =
		\frac{\beta^2_1}{-\alpha^3_1}F \nonumber \\
	\bar{\psi}_a = \frac{\psi_a}{\hat{u}_1} =
		\sqrt{\frac{\beta_1}{-\alpha_1}}\psi_a \nonumber \\
	\bar{A} = \frac{A}{\hat{u}_1} \\
	\bar{e} = \frac{1}{\sqrt{2}}\hat{u}_1\hat{\xi}_1 e^* =
		\frac{e^*}{\sqrt{\beta_1}} \nonumber \\
	\bar{\alpha}_2 = 2\hat{\xi}^2_1 \alpha_2 = \frac{\alpha_2}{-\alpha_1}
	  \nonumber \\
	\bar{\eta}_1 = 2\hat{\xi}^2_1 \eta_1 = \frac{\eta_1}{-\alpha_1} \nonumber \\
	\bar{\eta}_2 = 2\hat{\xi}^2_1 \hat{u}^2_1 \eta_2 =
		\frac{\eta_2}{\beta_1} \nonumber \\
	\bar{\nu} = \nu \label{eq:A-6}
\end{gather}

Подставляя их в \eqref{eq:2} получаем
\begin{align}
	\bar{F} = & \frac{1}{2}\abs{\left( \hat{\nabla} + i\hat{e}\hat{A} \right)
	  \hat{\psi}_1}^2 + \frac{1}{2}\abs{\left( \hat{\nabla} + i\hat{e}\hat{A}
	  \right)\hat{\psi}_2}^2 - \nu\Re\left\{ \left( \hat{\nabla} + i\hat{e}\hat{A}
	  \right) \hat{\psi}_1 \cdot \left( \hat{\nabla} - i\hat{e}\hat{A} \right)
		\hat{\psi}^*_2 \right\} + \nonumber \\
  & + \frac{1}{2}\abs{\hat{\nabla}\times\hat{A}}^2 - \abs{\hat{\psi}_1}^2 +
		\frac{1}{2}\abs{\hat{\psi}_1}^2 - \hat{\alpha_2}\abs{\hat{\psi}_2}^2 +
		\frac{\hat{\beta}_2}{2}\left| \hat{\psi}_2 \right|^2 - \nonumber \\
  & - \hat{\eta}_1\abs{\hat{\psi}_1}\abs{\hat{\psi}_2} \cos(\theta_1 - \theta_2)
    + \hat{\eta}_2\abs{\hat{\psi}_1}^2 \abs{\hat{\psi}_2}^2, \label{eq:A-7}
\end{align}

Это (опуская верхнюю черту) и есть энергия ГЛ которая
используется в разделе~\ref{ch:4} для целей численного моделирования.

Окончательно отметим, что однокомпонентная модель ГЛ получается
из \eqref{eq:A-7} определяя \( \psi_2 \equiv 0 \), глубину проникновения
\( \lambda = 1/\mu_a = 1/e \) и длину когерентности
\( \xi = 1/\sqrt{2} \), и, следовательно, параметр теории ГЛ
\begin{equation}
	\kappa = \lambda/\xi = \frac{\sqrt{2}}{e}
	\label{eq:A-8}
\end{equation}

Так, в параметрах используемых в разделе~\ref{ch:4}, можно считать \( e \) в виде
обратного параметра теории ГЛ для однозонной модели. Значение \( e \)
соответствующий предельной однокомпонентной теории Богомольного c
\( e_c = 2 \) в этой интерпретации.
