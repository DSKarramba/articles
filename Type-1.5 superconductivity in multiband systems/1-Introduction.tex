\chapter{Введение}
\label{ch:1}

According to Ginzburg-Landau theory, a conventional
superconductor near T c is described by a single complex
order parameter field. The physics of these systems is
governed by two fundamental length scales, the magnetic
field penetration depth \( \lambda \) and the coherence length \( \xi \), and
the ratio \( \kappa \) of these determines the response to an external
field, sorting them into two categories as follows; type-I
when \( \kappa < 1/\sqrt{2} \) and type-II when \( \kappa > 1/\sqrt{2} \)
\footnotemark[3] .
Type-I superconductors expel weak magnetic fields,
while strong fields give rise to formation of macroscopic
normal domains with magnetic flux 4 . The response of
type-II superconductors is completely different; below
some critical value \( H_{c1} \), the field is expelled. Above this
value a superconductor forms a lattice or a liquid of vor-
tices that have a supercurrent circulating around a nor-
mal core and carry magnetic flux through the system.
Finally, at a higher second critical value, \( H_{c2} \) supercon-
ductivity is destroyed.
These different responses are usually viewed as conse-
quences of the vortex interaction in these systems, the
energy cost of a boundary between superconducting and
normal states and the thermodynamic stability of vor-
tex excitations. In a type-II superconductor the energy
cost of a boundary between the normal and the supercon-
ducting state is negative, while the interaction between
vortices is repulsive 3 . This leads to a formation of stable
vortex lattices and liquids. In type-I superconductors the
situation is the opposite; the vortex interaction is attrac-
tive (thus making them unstable against collapse into one
large vortex), while the boundary energy between normal
and superconducting states is positive. From a thermo-
dynamic point of view the principal difference between
type-I and type-II states is the following: (i) In type-
II superconductors the external magnetic field strength
required to make formation of vortex excitations energet-
ically preferred, \( H_{c1} \), is smaller than the thermodynam-
ical magnetic field \( H_{ct} \) (the field whose energy density
is equal to the condensation energy of a superconduc-
tor, i.e. the field at which the uniform superconducting
state becomes thermodynamically unstable); (ii) In type-
I superconductors the field strength required to create
a vortex excitation is larger than the thermodynamical
critical magnetic field i.e. vortices cannot form. One can
distinguish also a special “zero measure” boundary case
where \( \kappa \) has a critical value exactly at the type-I/type-II
boundary, which in the most common GL model parame-
terization corresponds to \( \kappa = 1/\sqrt{2} \). In that case vortices
do not interact 5 in the Ginzburg-Landau theory.
The above circumstances result in a situation where, in
a strong external magnetic field, type-I superconductors
usually have a tendency to minimize boundary energy
between the normal and superconducting states, leading
to a formation of large inclusions of normal phase which
frequently have laminar structure 4 .
Recently there has been increased interest in supercon-
ductors with several superconducting components. The
main situations where multiple superconducting compo-
nents arise are (i) multiband superconductors 6 - 11 , (ii)
mixtures of independently conserved condensates such as
the projected superconductivity in metallic hydrogen and
hydrogen rich alloys 12–14 and (iii) superconductors with
other than s-wave pairing symmetries. In this work we
focus on the cases (i) and (ii). The principal difference
between the cases (i) and (ii) is the absence of the inter-
component Josephson coupling in case (ii).
In two-band superconductors (i) the superconducting
components originate from electronic Cooper pairing in
different bands 6 . Therefore these condensates could not a
priori be expected to be independently conserved. This,
at the level of effective models should manifest itself in
a rather generic presence of intercomponent Josephson
coupling.
In the case (ii) two superconducting components
were predicted to originate from electronic and pro-
tonic Cooper pairing in metallic hydrogen or hydrogen-
rich alloys. In the projected liquid metallic deuteriumor deuterium-rich alloys, electronic superconductivity
was predicted to coexist at ultra high pressures with
deuteronic condensation 12–14 . Because electrons cannot
be converted to protons or deuterons the condensates
are independently conserved, and therefore in the effec-
tive model intercomponent Josephson coupling is forbid-
den on symmetry grounds. These states are currently
a subject of a renewed experimental pursuit. They are
expected to arise at high but experimentally accessible
pressures (\( \approx 400 \)~GPa). Current static compression ex-
periments achieve pressures of \( \approx 350 \)~GPa with pressures
of an order of 1TPa being anticipated in diamond anvil
cell experiments due to the recent availability of ultra
hard diamonds. Similar two-charged component models
were discussed in the context of the physics of neutron
stars where they represent coexistent protonic and \( \Sigma^\text{--} \)-hyperon Cooper pairs in the neutron star interior 15 .
This wide variety of systems raises the need to un-
derstand and classify the possible magnetic responses of
multicomponent superconductors. It was discussed re-
cently that in multicomponent systems the magnetic re-
sponse is much more complex than in ordinary systems,
and that the type-I/type-II dichotomy is not sufficient
for classification. Rather, in a wide range of parameters,
as a consequence of the existence of three fundamental
length scales, there is a separate superconducting regime
where vortices have long-range attractive, short-range re-
pulsive interaction and form vortex clusters immersed in
domains of two-component Meissner state 1,2 . Recent ex-
perimental works 16,17 have put forward the suggestion
that this state is realized in the two-band material MgB 2 ,
which sparked growing interest in this topic. In particu-
lar questions were raised over whether this “type-1.5” su-
perconducting regime (as it was termed by Moshchalkov
et al 16 , for recent works see 18 ) is possible even in principle
in the case of various non vanishing couplings (e.g. in-
trinsic Josephson coupling, mixed gradient couplings etc)
between superconducting components in different bands.
In this work we report a study of the appearance
of type-1.5 superconductivity especially focusing on the
case of multiband superconductivity, demonstrating the
persistence of this type of superconductivity in the pres-
ence of various kinds of intercomponent couplings (such
as interband Josephson coupling, mixed gradient cou-
pling, density-density, and other kinds coupling).

\section{Сверхпроводимость типа 1,5}
\label{sec:1-1}

The possibility of a new type of superconductivity,
distinct from the type-I and type-II in multicomponent
systems 1,2 is based on the following considerations. In
principle the boundary problem in the Ginzburg-Landau
type of equations in the presence of phase winding is
not, from a rigorous point of view, reducible to a one-
dimensional problem in general. Furthermore, as dis-
cussed in 1,2 , in general in two-component models there
are three fundamental length scales which renders the

\begin{figure}[h!]
  \includegraphics[width=.5\textwidth]{1-01}
  \caption{A comparison of the magnetic phase diagrams of
    clean bulk type-I,type-II and type-1.5 superconductors at
    zero temperature. The semi-Meissner state is a macroscopic
    phase separation into two-component Meissner state and vortex
    clusters where one of the density modes is suppressed by
    core overlaps}
\end{figure}


model impossible to parametrize in terms of a single di-
mensionless parameter \( \kappa \). In the case where the conden-
sates are not coupled by interband Josephson coupling
but only by the vector potential these length scales are
the two independent coherence lengths (set by the in-
verse masses of the corresponding scalar density fields)
and magnetic field penetration length (set by the in-
verse mass acquired by the gauge field). In contrast,
in the case where the condensates are coupled by inter-
band Josephson terms, one cannot distinguish indepen-
dent coherence lengths attributed to different conden-
sates. Nonetheless, in this case the density variations
can also possess two fundamental length scales 2 , in con-
trast to single-component theories. We elaborate on this
fact below. In 1,2 vortex solutions in two-component the-
ories were found which have non-monotonic vortex inter-
action, with a long range attractive part determined by
a dominant density-density interaction and a short range
repulsive part produced by current-current and electro-
magnetic interactions. An important circumstance which
was demonstrated was that these vortices are thermody-
namically stable in spite of the existence of the attractive
tail in the interaction.
A non-monotonic intervortex interaction potential
should result in the formation of vortex clusters in low
magnetic field immersed into the vortexless areas, a state
referred to in 1 as the “semi-Meissner state”. Figure 1
shows the schematic phase diagram of a type-1.5 super-
conductor.
If the vortices form clusters one cannot use the
usual one-dimensional argument concerning the energy
of superconductor-to-normal state boundary to classify
the magnetic response of the system. First of all, the
energy per vortex in such a case depends on whether a
vortex is placed in a cluster or not: i.e. formation of a
single isolated vortex might be energetically unfavorable,
while formation of vortex clusters is favorable, because
in a cluster where vortices are placed in a minimum of
the interaction potential, the energy per flux quantum issmaller than that for an isolated vortex (thermodynam-
ically the nonmonotonic two-vortex interaction potential
predicts that the smallest energy per flux quantum will
be in the case of a uniform lattice with spacing equal to
the minimum of two-body intervortex potential).
Thus, besides the energy of a vortex in a cluster, there
appears an additional energy characteristic associated
with the boundary of a cluster. In other words, in this
situation, to determine the magnetic response of a system
it is not sufficient to study the one-dimensional bound-
ary problem nor the single-vortex problem, in contrast
to single component systems. Moreover, in a cluster
the system tends to minimize the boundary energy of
a cluster (similarly to type-I behavior), while breaking
into a lattice of one-quantum vortices inside the cluster
(similarly to type-II systems with negative interface en-
ergy). Thus, in an increased magnetic field the vortices
form via a first order phase transition. A magnetic phase
distinct from the vortex and Meissner states which then
arises is a macroscopic phase separation into domains of
two-component Meissner state and vortex clusters where
one of the density modes is suppressed by core overlap.
We summarize the basic properties of type-I, type-II and
type-1.5 regimes in the table I.
The existence of thermodynamically stable type-1.5 su-
perconducting regimes ultimately depends on the exis-
tence of a nonmonotonic intervortex interaction poten-
tial. It is an important question how generic this effect is.
In this work we mainly focus on multiband realizations
of multicomponent superconductivity and investigate the
effects of interband Josephson coupling, mixed gradient
coupling, and density-density coupling terms on vortex
interactions in two band superconductors. We show that
(i) when these couplings are present, the system still can
possess three fundamental length scales, in contrast to
the two length scales in the usual single-component GL
theory; (ii) non-monotonic interaction is possible in a
wide parameter range in these models.
The structure of this paper is as follows: In section II
we introduce the model.
In section III we present a linear theory of asymptotics
of the vortex fields in a superconductor with two bands
with various interband couplings.
We begin section III by demonstrating that for a gen-
eral form of the effective potential in a two-band (or more
generally two-gap) Ginzburg-Landau free energy, the lin-
ear theory gives, under quite general conditions, two fun-
damental length scales of the variations of the densities.
From the linearized theory we calculate the long-range in-
tervortex interaction potentials using the two-component
generalization of the point-vortex method 19 and show
how the non-monotonic intervortex interaction potential
arise from the interplay of two fundamental length scales
of the superfluid density variations and the magnetic field
penetration length. The central point of this part is how
the fundamental length scales are defined in the presence
of interband coupling as well as the occurrence of “mode
mixing”. Next we move to quantitatively study the ef-
fects of several kinds of intercomponent couplings which
quite generically arise in two-component theories.
In section III (d) we demonstrate that that mixed gra-
dient coupling can lead under certain conditions to an
increase in the disparity of the characteristic scales of
the density variations.
In Section IV we present a large scale numerical study
of the full nonlinear problem of the interaction between
a pair of vortices.
