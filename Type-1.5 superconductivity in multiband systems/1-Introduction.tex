\chapter{Введение}
\label{ch:1}

Согласно теории Гинзбурга-Ландау обычно сверхпроводник вблизи \( T_c \) 
описывается одним комплексным параметром поля. Физика этих систем определяется 
двумя фундаментальными масштабными длинами, глубиной проникновения магнитного 
поля \( \lambda \) и длиной когерентности \( \xi \), а также коэффициентом 
\( \kappa \), который определяет реакцию на внешнее поле, разделяя их на 
категории следующим образом; сверхпроводники первого рода, где 
\( \kappa < 1/\sqrt{2} \) и второго рода, где \( \kappa > 1/\sqrt{2} \) 
\cite{bib:3}.

Свехрпроводники первого рода исключают слабые магнитные поля, в то время как 
сильные поля порождают формирования макроскопически нормальных областей с 
магнитным потоком \cite{bib:4}. Реакция сверхпроводников второго рода 
совершенно иная; ниже некоторого критического значения \( H_{c1} \), поле 
выталкивается. 
Above 
this value a superconductor forms a lattice or a liquid of vortices that have a 
supercurrent circulating around a normal core and carry magnetic flux through 
the system. 
И, наконец, при значении больше второго критического, \( H_{c2} \) 
сверхпроводимость разрушается. Эти различные ответы, как правило, 
рассматривается как последствия взаимодействия вихрей в этих системах, расход 
энергии на границе между сверхпроводящем и нормальном состояниях и 
термодинамической устойчивости вихревых возбуждений (!). В сверхпроводника 
второго рода расход энергии на границе между нормальной и сверхпроводящем 
состоянием является отрицательным, а взаимодействие между вихрями является 
отталкивающим.\cite{bib:3}. Это приводит к образованию устойчивых вихревых 
решеток и liquids. В сверхпроводниках первого рода ситуация противоположная; 
вихрь взаимодействия притягивающий (что делает их неустойчивыми против распада 
в один большой вихрь), а граница энергии между нормальным и сверхпроводящим 
состояниях положительно. С термодинамической точки зрения принципиальное 
отличие сверхпроводников первого рода от второго следующее: (i) В 
сверхпроводниках второго рода напряженности внешнего магнитного поля, 
необходимая, чтобы сделать образование вихревых возбуждений энергетически 
выгодными, \( H_{c1} \), меньше, чем термодинамическое магнитного поля 
\( H_{ct} \) (поле, энергетическая плотность равна энергии конденсации 
сверхпроводника, т.е. области, в которой равномерное сверхпроводящее состояние 
становится термодинамически неустойчивым); (ii) В сверхпроводниках первого рода 
напряженность поля требуется создание возбуждение вихря больше, чем 
критическое термодинамическое магнитное поле, т.е. вихри не могут 
образовываться. Можно выделить также специальный "нульмерный" пограничный 
случай, когда \( \kappa \) имеет критическое значение точно на границе 
первого/второго рода, что в самом общей модели ГЛ соответствует 
\( \kappa = 1/\sqrt{2} \). В этом случае вихри не взаимодействуют\cite{bib:5} 
в теории Гинзбурга-Ландау.

Вышеуказанные обстоятельства приводят к тому, что, в сильном внешнем магнитном 
поле, сверхпроводники первого рода обычно имеют тенденцию к минимизации 
энергии на границе между нормальной и сверхпроводящей фазой, что приводит к 
образованию крупных включений нормальной фазы, которые часто имеют слоистую 
структуру\cite{bib:4}. 

Recently there has been increased interest in superconductors with several 
superconducting components. The main situations where multiple superconducting 
components arise are (i) multiband superconductors
\cite{bib:6,bib:7,bib:8,bib:9,bib:10,bib:11}, 
(ii) mixtures of independently conserved condensates such as the projected 
superconductivity in metallic hydrogen and hydrogen rich alloys 
\cite{bib:12,bib:13,bib:14} and (iii) superconductors with other than s-wave 
pairing symmetries. In this work we focus on the cases (i) and (ii). The 
principal difference between the cases (i) and (ii) is the absence of the 
inter-component Josephson coupling in case (ii). 

In two-band superconductors (i) the superconducting components originate from 
electronic Cooper pairing in different bands\cite{bib:6}. Therefore these 
condensates could not a priori be expected to be independently conserved. 
This, at the level of effective models should manifest itself in a rather 
generic presence of intercomponent Josephson coupling. 

In the case (ii) two superconducting components were predicted to originate 
from electronic and protonic Cooper pairing in metallic hydrogen or 
hydrogen-rich alloys. In the projected liquid metallic deuteriumor 
deuterium-rich alloys, electronic superconductivity was predicted to coexist 
at ultra high pressures with deuteronic condensation 
\cite{bib:12,bib:13,bib:14}. Because electrons cannot be converted to protons 
or deuterons the condensates are independently conserved, and therefore in the 
effective model intercomponent Josephson coupling is forbidden on symmetry 
grounds. These states are currently a subject of a renewed experimental 
pursuit. They are expected to arise at high but experimentally accessible 
pressures (\( \approx 400 \)~GPa). Current static compression experiments 
achieve pressures of \( \approx 350 \)~GPa with pressures of an order of 
1TPa being anticipated in diamond anvil cell experiments due to the recent 
availability of ultra hard diamonds. Similar two-charged component models 
were discussed in the context of the physics of neutron stars where they 
represent coexistent protonic and \( \Sigma^\text{--} \)-hyperon Cooper pairs 
in the neutron star interior\cite{bib:15}. 

This wide variety of systems raises the need to understand and classify the 
possible magnetic responses of multicomponent superconductors. It was 
discussed recently that in multicomponent systems the magnetic response is 
much more complex than in ordinary systems, and that the type-I/type-II 
dichotomy is not sufficient for classification. Rather, in a wide range of 
parameters, as a consequence of the existence of three fundamental length 
scales, there is a separate superconducting regime where vortices have 
long-range attractive, short-range repulsive interaction and form vortex 
clusters immersed in domains of two-component Meissner state 
\cite{bib:1,bib:2}. Recent experimental works \cite{bib:16,bib:17} have put 
forward the suggestion that this state is realized in the two-band material 
\( MgB_2 \), which sparked growing interest in this topic. In particular 
questions were raised over whether this "type-1.5" superconducting regime (as 
it was termed by Moshchalkov et al\cite{bib:16}, for recent works see
\cite{bib:18}) is possible even in principle in the case of various non 
vanishing couplings (e.g. intrinsic Josephson coupling, mixed gradient 
couplings etc) between superconducting components in different bands. 

In this work we report a study of the appearance of type-1.5 superconductivity 
especially focusing on the case of multiband superconductivity, demonstrating 
the persistence of this type of superconductivity in the presence of various 
kinds of intercomponent couplings (such as interband Josephson coupling, mixed 
gradient coupling, density-density, and other kinds coupling).

\section{Сверхпроводимость типа 1,5}
\label{sec:1-1}

The possibility of a new type of superconductivity, distinct from the type-I 
and type-II in multicomponent systems \cite{bib:1,bib:2} is based on the 
following considerations. In principle the boundary problem in the 
Ginzburg-Landau type of equations in the presence of phase winding is not, 
from a rigorous point of view, reducible to a one-dimensional problem in 
general. Furthermore, as discussed in \cite{bib:1,bib:2}, in general in 
two-component models there are three fundamental length scales which renders 
the

\begin{figure}[h!]
  \includegraphics[width=.5\textwidth]{1-01}
  \caption{A comparison of the magnetic phase diagrams of
    clean bulk type-I,type-II and type-1.5 superconductors at
    zero temperature. The semi-Meissner state is a macroscopic
    phase separation into two-component Meissner state and vortex
    clusters where one of the density modes is suppressed by
    core overlaps}
  \label{fig:1}
\end{figure}


model impossible to parametrize in terms of a single dimensionless parameter 
\( \kappa \). In the case where the condensates are not coupled by interband 
Josephson coupling but only by the vector potential these length scales are
the two independent coherence lengths (set by the inverse masses of the 
corresponding scalar density fields) and magnetic field penetration length 
(set by the inverse mass acquired by the gauge field). In contrast, in the 
case where the condensates are coupled by inter-band Josephson terms, one 
cannot distinguish independent coherence lengths attributed to different 
condensates. Nonetheless, in this case the density variations can also possess 
two fundamental length scales\cite{bib:2}, in contrast to single-component 
theories. We elaborate on this fact below. In \cite{bib:1,bib:2} vortex 
solutions in two-component theories were found which have non-monotonic 
vortex inter-action, with a long range attractive part determined by a 
dominant density-density interaction and a short range repulsive part produced 
by current-current and electro-magnetic interactions. An important 
circumstance which was demonstrated was that these vortices are 
thermodynamically stable in spite of the existence of the attractive tail in 
the interaction.

A non-monotonic intervortex interaction potential should result in the 
formation of vortex clusters in low magnetic field immersed into the 
vortexless areas, a state referred to in \cite{bib:1} as the "semi-Meissner 
state". Figure \ref{fig:1} shows the schematic phase diagram of a type-1.5 
superconductor.

If the vortices form clusters one cannot use the usual one-dimensional 
argument concerning the energy of superconductor-to-normal state boundary to 
classify the magnetic response of the system. First of all, the energy per 
vortex in such a case depends on whether a vortex is placed in a cluster or 
not: i.e. formation of a single isolated vortex might be energetically 
unfavorable, while formation of vortex clusters is favorable, because in a 
cluster where vortices are placed in a minimum of the interaction potential, 
the energy per flux quantum issmaller than that for an isolated vortex 
(thermodynamically the nonmonotonic two-vortex interaction potential predicts 
that the smallest energy per flux quantum will be in the case of a uniform 
lattice with spacing equal to the minimum of two-body intervortex potential).

Thus, besides the energy of a vortex in a cluster, there appears an additional 
energy characteristic associated with the boundary of a cluster. In other 
words, in this situation, to determine the magnetic response of a system it is 
not sufficient to study the one-dimensional boundary problem nor the 
single-vortex problem, in contrast to single component systems. Moreover, in a 
cluster the system tends to minimize the boundary energy of a cluster 
(similarly to type-I behavior), while breaking into a lattice of one-quantum 
vortices inside the cluster (similarly to type-II systems with negative 
interface energy). Thus, in an increased magnetic field the vortices form via 
a first order phase transition. A magnetic phase distinct from the vortex and 
Meissner states which then arises is a macroscopic phase separation into 
domains of two-component Meissner state and vortex clusters where one of the 
density modes is suppressed by core overlap. We summarize the basic properties 
of type-I, type-II and type-1.5 regimes in the table I.

The existence of thermodynamically stable type-1.5 superconducting regimes 
ultimately depends on the existence of a nonmonotonic intervortex interaction 
potential. It is an important question how generic this effect is. In this 
work we mainly focus on multiband realizations of multicomponent 
superconductivity and investigate the effects of interband Josephson coupling, 
mixed gradient coupling, and density-density coupling terms on vortex 
interactions in two band superconductors. We show that (i) when these 
couplings are present, the system still can possess three fundamental length 
scales, in contrast to the two length scales in the usual single-component GL 
theory; (ii) non-monotonic interaction is possible in a wide parameter range 
in these models.

The structure of this paper is as follows: In section II we introduce the 
model.In section III we present a linear theory of asymptotics of the vortex 
fields in a superconductor with two bands with various interband couplings.

We begin section III by demonstrating that for a general form of the effective 
potential in a two-band (or more generally two-gap) Ginzburg-Landau free 
energy, the linear theory gives, under quite general conditions, two 
fundamental length scales of the variations of the densities. From the 
linearized theory we calculate the long-range intervortex interaction 
potentials using the two-component generalization of the point-vortex method 
\cite{bib:19} and show how the non-monotonic intervortex interaction potential 
arise from the interplay of two fundamental length scales of the superfluid 
density variations and the magnetic field penetration length. The central 
point of this part is how the fundamental length scales are defined in the 
presence of interband coupling as well as the occurrence of "mode mixing". 
Next we move to quantitatively study the effects of several kinds of 
intercomponent couplings which quite generically arise in two-component 
theories.

In section III (d) we demonstrate that that mixed gradient coupling can lead 
under certain conditions to an increase in the disparity of the characteristic 
scales of the density variations.

В разделе IV мы представляем большое численное исследование полной нелинейной 
задачи взаимодействия между парой вихрей.
