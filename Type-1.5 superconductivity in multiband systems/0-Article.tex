\documentclass[notitlepage,12pt,a4paper,draft]{report}
\usepackage[utf8]{inputenc}
\usepackage[russian]{babel}
\usepackage[margin=2cm]{geometry}

\usepackage[vectors,derivative,complex,root]{hedmaths}
\usepackage[column]{hedfeatures}

\usepackage{pscyr}

\usepackage{graphicx}
\graphicspath{{images/}}

\usepackage[usenames,dvipsnames]{color}
\usepackage[colorlinks,linkcolor=black,citecolor=black,urlcolor=Blue]{hyperref}

\newcommand{\E}{\mathcal{E}}
\newcommand{\h}{\mathcal{H}}

\addto\captionsrussian{\renewcommand\bibname{Ссылки}}
\usepackage{cite}

\makeatletter   
\renewcommand\@biblabel[1]{\textsuperscript{#1}}

\def\@cite#1#2{\leavevmode\cite@adjust%
  \textsuperscript{#1\if@tempswa\@safe@activesfalse\citemid{#2}\fi
  \spacefactor\@m
  }
 \@restore@auxhandle}

\renewcommand{\@makechapterhead}[1]{%
  {\parindent \z@ \raggedright \normalfont
    \ifnum \c@secnumdepth >\m@ne
        \Large\bfseries \@chapapp\space \thechapter.\space
    \fi
    \Large \bfseries #1\par\nobreak
    \vskip 3ex
  }}
\renewcommand{\@makeschapterhead}[1]{%
  {\parindent \z@ \raggedright
    \normalfont
    \interlinepenalty\@M
    \Large \bfseries  #1\par\nobreak
    \vskip 3ex
  }}

\renewcommand\section{\@startsection      {section}{1}{\z@}{3ex}{3ex}%
                                          {\normalfont\large\bfseries}}
\renewcommand\subsection{\@startsection   {subsection}{2}{\z@}{2ex}{2ex}%
                                          {\normalfont\normalsize\bfseries}}
\renewcommand\subsubsection{\@startsection{subsubsection}{3}{\z@}{1.5ex}{1.5ex}%
                                          {\normalfont\normalsize}}
\renewcommand\paragraph{\@startsection    {paragraph}{4}{\z@}{1ex}{-1em}
                                          {\normalfont\normalsize}}
\renewcommand\subparagraph{\@startsection {subparagraph}{5}{\parindent}{1ex}{-1em}%
                                          {\normalfont\normalsize}}
\makeatother

\righthyphenmin=2

\begin{document}
  
  \begin{center}
    \begin{minipage}{.8\textwidth} 
    \center  
    \textbf{Сверхпроводимость типа 1,5 в многозонных системах: эффекты межзонных
      соединений}

    Йохан Карлстром, Егор Бабаев и Мартин Спейт
    \end{minipage}
    
    \bigskip
    \begin{minipage}{.95\textwidth}      
      \hspace{2em} В отличие от однокомпонентных сверхпроводников, описываемых на
      уровне теории Гинзбурга-Ландау одним параметром \( \kappa \) и разделенных
      на сверхпроводники I~рода (при \( \kappa < 1/\sqrt{2} \)) и
      сверхпроводники II~рода (при \( \kappa > 1/\sqrt{2} \)),
      двух\-ком\-по\-нент\-ные системы, в общем, тремя основными масштабными
      величинами и, как оказалось, обладают отдельным сверхпроводящим состоянием
      <<типа 1,5>>\cite{bib:1,bib:2}. В этом состоянии, вследствие наличия
      дополнительной фундаментальной масштабной величины, вихри притягиваются
      друг к другу на больших расстояниях, но отталкиваются на более малых, и,
      следовательно, должны образовывать кластеры в слабых магнитных полях.
      В этой статье мы исследуем появления сверхпроводящего состояния типа 1,5 и
      интерпретируем основные масштабные величины в случае двух зон и важные
      межзонные взаимодействия: внутренний Джозефсоновский контакт, смешанный
      градиент спаривания и взаимодействия плотность-плотность. Мы покажем, что
      при наличии таких межкомпонентных взаимодействий система поддерживает
      сверхпроводимость типа 1,5 с фундаментальной масштабной величиной,
      связанной с массой калибровочного поля и двумя массами нормальных режимов,
      представленных смешанными комбинациями полей плотности.
    \end{minipage}
  \end{center}
  
  \chapter{Введение}
\label{ch:1}

According to Ginzburg-Landau theory, a conventional superconductor near 
\( T \) c is described by a single complex order parameter field. The physics 
of these systems is governed by two fundamental length scales, the magnetic
field penetration depth \( \lambda \) and the coherence length \( \xi \), and
the ratio \( \kappa \) of these determines the response to an external
field, sorting them into two categories as follows; type-I when 
\( \kappa < 1/\sqrt{2} \) and type-II when \( \kappa > 1/\sqrt{2} \)
\footnotemark[3].

Type-I superconductors expel weak magnetic fields, while strong fields give 
rise to formation of macroscopic normal domains with magnetic flux 4. The 
response of type-II superconductors is completely different; below some 
critical value \( H_{c1} \), the field is expelled. Above this value a 
superconductor forms a lattice or a liquid of vortices that have a 
supercurrent circulating around a normal core and carry magnetic flux through 
the system. Finally, at a higher second critical value, \( H_{c2} \) 
superconductivity is destroyed. 

These different responses are usually viewed as consequences of the vortex 
interaction in these systems, the energy cost of a boundary between 
superconducting and normal states and the thermodynamic stability of vortex 
excitations. In a type-II superconductor the energy cost of a boundary between 
the normal and the superconducting state is negative, while the interaction 
between vortices is repulsive 3. This leads to a formation of stable vortex 
lattices and liquids. In type-I superconductors the situation is the opposite; 
the vortex interaction is attractive (thus making them unstable against 
collapse into one large vortex), while the boundary energy between normal and 
superconducting states is positive. From a thermodynamic point of view the 
principal difference between type-I and type-II states is the following: (i) 
In type-II superconductors the external magnetic field strength required to 
make formation of vortex excitations energetically preferred, \( H_{c1} \), 
is smaller than the thermodynamical magnetic field \( H_{ct} \) (the field 
whose energy density is equal to the condensation energy of a superconductor, 
i.e. the field at which the uniform superconducting state becomes 
thermodynamically unstable); (ii) In type-I superconductors the field strength 
required to create a vortex excitation is larger than the thermodynamical 
critical magnetic field i.e. vortices cannot form. One can distinguish also a 
special "zero measure" boundary case where \( \kappa \) has a critical value 
exactly at the type-I/type-II boundary, which in the most common GL model 
parameterization corresponds to \( \kappa = 1/\sqrt{2} \). In that case 
vortices do not interact 5 in the Ginzburg-Landau theory. 

The above circumstances result in a situation where, in a strong external 
magnetic field, type-I superconductors usually have a tendency to minimize 
boundary energy between the normal and superconducting states, leading to a 
formation of large inclusions of normal phase which frequently have laminar 
structure \( ^4 \). 

Recently there has been increased interest in superconductors with several 
superconducting components. The main situations where multiple superconducting 
components arise are (i) multiband superconductors 6 - 11 , (ii) mixtures of 
independently conserved condensates such as the projected superconductivity 
in metallic hydrogen and hydrogen rich alloys 12–14 and (iii) superconductors 
with other than s-wave pairing symmetries. In this work we focus on the cases 
(i) and (ii). The principal difference between the cases (i) and (ii) is the 
absence of the inter-component Josephson coupling in case (ii). 

In two-band superconductors (i) the superconducting components originate from 
electronic Cooper pairing in different bands 6. Therefore these condensates 
could not a priori be expected to be independently conserved. This, at the 
level of effective models should manifest itself in a rather generic presence 
of intercomponent Josephson coupling. 

In the case (ii) two superconducting components were predicted to originate 
from electronic and protonic Cooper pairing in metallic hydrogen or 
hydrogen-rich alloys. In the projected liquid metallic deuteriumor 
deuterium-rich alloys, electronic superconductivity was predicted to coexist 
at ultra high pressures with deuteronic condensation 12–14 . Because electrons 
cannot be converted to protons or deuterons the condensates are independently 
conserved, and therefore in the effective model intercomponent Josephson 
coupling is forbidden on symmetry grounds. These states are currently a 
subject of a renewed experimental pursuit. They are expected to arise at high 
but experimentally accessible pressures (\( \approx 400 \)~GPa). Current 
static compression experiments achieve pressures of \( \approx 350 \)~GPa 
with pressures of an order of 1TPa being anticipated in diamond anvil cell 
experiments due to the recent availability of ultra hard diamonds. Similar 
two-charged component models were discussed in the context of the physics of 
neutron stars where they represent coexistent protonic and 
\( \Sigma^\text{--} \)-hyperon Cooper pairs in the neutron star 
interior\( ^{15} \). 

This wide variety of systems raises the need to understand and classify the 
possible magnetic responses of multicomponent superconductors. It was 
discussed recently that in multicomponent systems the magnetic response is 
much more complex than in ordinary systems, and that the type-I/type-II 
dichotomy is not sufficient for classification. Rather, in a wide range of 
parameters, as a consequence of the existence of three fundamental length 
scales, there is a separate superconducting regime where vortices have 
long-range attractive, short-range repulsive interaction and form vortex 
clusters immersed in domains of two-component Meissner state 1,2. Recent 
experimental works 16,17 have put forward the suggestion that this state is 
realized in the two-band material \( MgB_2 \), which sparked growing interest 
in this topic. In particular questions were raised over whether this 
"type-1.5" superconducting regime (as it was termed by Moshchalkov et 
al\( ^{16} \), for recent works see 18 ) is possible even in principle in the 
case of various non vanishing couplings (e.g. intrinsic Josephson coupling, 
mixed gradient couplings etc) between superconducting components in different 
bands. 

In this work we report a study of the appearance of type-1.5 superconductivity 
especially focusing on the case of multiband superconductivity, demonstrating 
the persistence of this type of superconductivity in the presence of various 
kinds of intercomponent couplings (such as interband Josephson coupling, mixed 
gradient coupling, density-density, and other kinds coupling).

\section{Сверхпроводимость типа 1,5}
\label{sec:1-1}

The possibility of a new type of superconductivity, distinct from the type-I 
and type-II in multicomponent systems 1,2 is based on the following 
considerations. In principle the boundary problem in the Ginzburg-Landau type 
of equations in the presence of phase winding is not, from a rigorous point of 
view, reducible to a one-dimensional problem in general. Furthermore, as 
discussed in 1,2 , in general in two-component models there are three 
fundamental length scales which renders the

\begin{figure}[h!]
  \includegraphics[width=.5\textwidth]{1-01}
  \caption{A comparison of the magnetic phase diagrams of
    clean bulk type-I,type-II and type-1.5 superconductors at
    zero temperature. The semi-Meissner state is a macroscopic
    phase separation into two-component Meissner state and vortex
    clusters where one of the density modes is suppressed by
    core overlaps}
\end{figure}


model impossible to parametrize in terms of a single dimensionless parameter 
\( \kappa \). In the case where the condensates are not coupled by interband 
Josephson coupling but only by the vector potential these length scales are
the two independent coherence lengths (set by the inverse masses of the 
corresponding scalar density fields) and magnetic field penetration length 
(set by the inverse mass acquired by the gauge field). In contrast, in the 
case where the condensates are coupled by inter-band Josephson terms, one 
cannot distinguish independent coherence lengths attributed to different 
condensates. Nonetheless, in this case the density variations can also possess 
two fundamental length scales 2 , in contrast to single-component theories. 
We elaborate on this fact below. In 1,2 vortex solutions in two-component 
theories were found which have non-monotonic vortex inter-action, with a long 
range attractive part determined by a dominant density-density interaction and 
a short range repulsive part produced by current-current and electro-magnetic 
interactions. An important circumstance which was demonstrated was that these 
vortices are thermodynamically stable in spite of the existence of the 
attractive tail in the interaction.

A non-monotonic intervortex interaction potential should result in the 
formation of vortex clusters in low magnetic field immersed into the 
vortexless areas, a state referred to in 1 as the "semi-Meissner state". 
Figure 1 shows the schematic phase diagram of a type-1.5 superconductor.

If the vortices form clusters one cannot use the usual one-dimensional 
argument concerning the energy of superconductor-to-normal state boundary to 
classify the magnetic response of the system. First of all, the energy per 
vortex in such a case depends on whether a vortex is placed in a cluster or 
not: i.e. formation of a single isolated vortex might be energetically 
unfavorable, while formation of vortex clusters is favorable, because in a 
cluster where vortices are placed in a minimum of the interaction potential, 
the energy per flux quantum issmaller than that for an isolated vortex 
(thermodynamically the nonmonotonic two-vortex interaction potential predicts 
that the smallest energy per flux quantum will be in the case of a uniform 
lattice with spacing equal to the minimum of two-body intervortex potential).

Thus, besides the energy of a vortex in a cluster, there appears an additional 
energy characteristic associated with the boundary of a cluster. In other 
words, in this situation, to determine the magnetic response of a system it is 
not sufficient to study the one-dimensional boundary problem nor the 
single-vortex problem, in contrast to single component systems. Moreover, in a 
cluster the system tends to minimize the boundary energy of a cluster 
(similarly to type-I behavior), while breaking into a lattice of one-quantum 
vortices inside the cluster (similarly to type-II systems with negative 
interface energy). Thus, in an increased magnetic field the vortices form via 
a first order phase transition. A magnetic phase distinct from the vortex and 
Meissner states which then arises is a macroscopic phase separation into 
domains of two-component Meissner state and vortex clusters where one of the 
density modes is suppressed by core overlap. We summarize the basic properties 
of type-I, type-II and type-1.5 regimes in the table I.

The existence of thermodynamically stable type-1.5 superconducting regimes 
ultimately depends on the existence of a nonmonotonic intervortex interaction 
potential. It is an important question how generic this effect is. In this 
work we mainly focus on multiband realizations of multicomponent 
superconductivity and investigate the effects of interband Josephson coupling, 
mixed gradient coupling, and density-density coupling terms on vortex 
interactions in two band superconductors. We show that (i) when these 
couplings are present, the system still can possess three fundamental length 
scales, in contrast to the two length scales in the usual single-component GL 
theory; (ii) non-monotonic interaction is possible in a wide parameter range 
in these models.

The structure of this paper is as follows: In section II we introduce the 
model.In section III we present a linear theory of asymptotics of the vortex 
fields in a superconductor with two bands with various interband couplings.

We begin section III by demonstrating that for a general form of the effective 
potential in a two-band (or more generally two-gap) Ginzburg-Landau free 
energy, the linear theory gives, under quite general conditions, two 
fundamental length scales of the variations of the densities. From the 
linearized theory we calculate the long-range intervortex interaction 
potentials using the two-component generalization of the point-vortex method 
19 and show how the non-monotonic intervortex interaction potential arise from 
the interplay of two fundamental length scales of the superfluid density 
variations and the magnetic field penetration length. The central point of 
this part is how the fundamental length scales are defined in the presence of 
interband coupling as well as the occurrence of "mode mixing". Next we move to 
quantitatively study the effects of several kinds of intercomponent couplings 
which quite generically arise in two-component theories.

In section III (d) we demonstrate that that mixed gradient coupling can lead 
under certain conditions to an increase in the disparity of the characteristic 
scales of the density variations.

In Section IV we present a large scale numerical study of the full nonlinear 
problem of the interaction between a pair of vortices.
 \newpage
  \chapter{Модель}
\label{ch:2}

\section{Функционал свободной энергии}
\label{sec:2-1}

Тип 1.5 изучается с помощью следующей двухкомпонентного функционала 
свободной энергии Гинзбурга-Ландау (ТГЛ):
\begin{align}
	F = \frac{1}{2}(D\psi_1)(D\psi_1)* + \frac{1}{2}(D\psi_2)(D\psi_2)* - 
		\nu Re\left( (D\psi_1)(D\psi_2)* \right) + 
		\frac{1}{2}\left(\nabla\times\vec{A}\right)^2 + F_p
	\label{eq:1}
\end{align}

Здесь \( D = \nabla + ie\vec{A} \) и \( \psi_a = |\psi_a|e^{i\theta_a} \), 
\( a = 1,2 \), представляет собой две сверхтекучих компоненты, которые в 
двухзонном сверхпроводнике соответствуют двум сверхтекучим densities в 
в различных диапазонах. Слагаемое \( F_p \) может содержать в нашем 
анализе произвольный набор не градиентных членов.

Особая форма двухкомпонентной модели ГЛ точно выведенная\cite{bib:8,bib:9,bib:10} для 
двухзонных сверхпроводников является:
\begin{gather}
	F = \frac{1}{2}(D\psi_1)(D\psi_1)* + \frac{1}{2}(D\psi_2)(D\psi_2)* - 
		\nu Re\left( (D\psi_1)(D\psi_2)* \right) + 
		\frac{1}{2}\left(\nabla\times\vec{A}\right)^2 + \alpha_1|\psi_1|^2 + 
		\frac{1}{2}\beta_1|\psi_1|^4 + \alpha_2|\psi_2|^2 + 
		\frac{1}{2}\beta_2|\psi_2|^4 + \eta_1|\psi_1||\psi_2|
		\cos(\theta_1-\theta_2) + \eta_2|\psi_1|^4|\psi_2|^2
	\label{eq:2}
\end{gather}

Первые два слагаемых представляют стандартный градиентный член Гинзбурга-Ландау, 
второе слагаемое представляет смешанные градиентные взаимодействия, которые 
появляются в двухзонных сверхпроводниках с примесным рассеянием
\cite{bib:8,bib:9}. Следующее слагаемое 

\section{Основные свойства вихревых возбуждений}
\label{sec:2-2}
 \newpage
  \chapter{Вихревая асимптотика}
\label{ch:3}

\section{Симметричная модель \texorpdfstring{$ U(1) \times U(1) $}
  {U(1) x U(1)}}
\label{sec:3-1}

\section{Джозефоновский контакт}
\label{sec:3-2}

\subsection{Сравнение со случаем пассивной второй band (?)}
\label{sec:3-3}

\section{Контакт плотность-плотность (?)}
\label{sec:3-4}

\section{Условия смешанного градиента}
\label{sec:3-5}
 \newpage
  \chapter{Численное решение нелинейной задачи}

\section{Weak Josephson coupling to a passive band}

\section{Effects of Josephson and mixed gradient terms in case of two active
  bands}

\section{Solutions with large disparity in the characteristic length scales}
 \newpage
  \chapter{Выводы}

В этой статье мы представили аналитическое и численное исследование о 
появлении сверхпроводимости типа 1.5 в случае двух зон с различными видами 
существенных межзонных соединений. Во всех случаях, которые мы рассматривали 
мы продемонстрировали, что система обладает тремя основными масштабами длин: 
первая \( 1/\mu_A \) связана с Лондоновской глубиной 
проникновения магнитного поля, в то время как остальные две \( 1/\mu_{1,2} \) 
связаны с характеристической масштабов длин ответственные за изменением 
глубины проникновения поля. В пределе двух конденсатов(?) связанных только с 
электромагнитными масштабными длинами \( 1/\mu_{1,2} \) с независимыми  
длинами когерентности двух полей. Однако мы показали, что введение ненулевой 
Джозефсоновской и density-density связи делают напряжённости полей спадающими 
по экспоненциальному закону при очень больших расстояниях от ядра, в то же 
время система всё ещё обладает двумя основными length scale, которые связаны 
с линейной комбинации напряженности полей повернутыми на <<угол смешивания>>. 
Третья основная масштабная длина в этом режиме является Лондоновской глубиной 
проникновения, и таким образом, двухзонная система со связью позволяет точно 
определить поведения типа 1.5. Далее мы изучали влияние смешанных градиентных 
членов и показали, как тип 1.5 описывается в этом случае. Мы показали, что в 
случае значительного смешанного градиента связи из определения трёх основных 
масштабных длин требуется дополнительные условия, поскольку он даёт режим 
смешивания, который не может быть описан одним углов смешивания. Важно 
отметить нами показанное, что смешанный градиент спаривания может увеличить 
несоответствие характерных масштабных длин изменения плотности. Можно провести 
аналогию между этим механизмом и механизмом качелей в физике нейтрино. Во 
второй части статьи мы представили сравнительное численное исследование 
вихрей типа 1.5 в различных режимах с различными межкомпонентными связями. 
Результаты были продемонстрированы в структуры двухкомпонентной модели 
Гинзбурга-Ландау с локальной электродинамикой. Однако мы ожидаем, что 
описание поведения типа 1.5 подобно присутствию в низких температурных 
режимах и в двухкомпонентных моделях с нелокальной электродинамикой.   

Понятие сверхпроводимости типа 1.5 можно непосредственно обобщить и в 
\( N \)-компонентном случае. Там может иметь место система с  
характерными масштабными длинами 
\( 
	\psi_1, \ldots, \psi_k < \lambda < \psi_{k+1}, \ldots, \psi_N
\) 
и есть термодинамически устойчивые вихри с немонотонным воздействием. 

Кроме многозонных сверхпроводников и сопутствующих электронных и ядерных 
сверхпроводников наша модель может быть реализована в искусственно созданных 
слоистых структурах из материалов типа-I и типа-II, где можно контролировать 
и настраивать межкомпонентную Джозефсоновскую связь.

\emph{Добавленное примечение:} После завершения этой работы, она была проверена в 
детальном расчёте, который не предполагает \( (1-T/T_C \) расширения, что 
модель Гинзбурга-Ландау \eqref{eq:2} довольно точно описывает физику вихрей 
двухзонных сверхпроводников в достаточно широком диапазоне параметров и 
температур\cite{bib:25}. \newpage
  \begin{appendix}
    \chapter{Единицы измерения}

В этом разделе мы дадим явное отображение нашего представления модели 
ГЛ к более общему виду представленного в учебниках. Рассмотрим модель 
Гинзбурга-Ландау в общем представлении
\begin{align*}
	F = \frac{\hbar^2}{2m_1}\left| 
		\left( \nabla - i\frac{e*}{\hbar c}A \right)\psi_1\right|^2 + 
		\frac{\hbar^2}{2m_2}\left| 
		\left( \nable - i\frac{e*}{\hbar c}A \right)\psi_2\right|^2 - 
		\nu\hbar^2 Re\left\{ \left( \nabla - i\frac{e*}{\hbac c} \rigth)\psi_1 
		\cdot\left( \nabla + i\frac{e*}{\hbar c}A \right)\psi*_2\right\} + \\
		+ \frac{1}{8\pi}\left( \nabla\times A \right^2 + 
		\alpha_1\left| \psi_1 \right|^2 + 
		\frac{1}{2}\beta_1\left| \psi_1 \right|^4 + 
		\alpha_2\left| \psi_2 \right|^2 + 
		\frac{1}{2}\beta_2\left| \psi_2 \right|^4 - \\
		-\eta_1\left| \psi_1 \right|\left| \psi_2 \right|
		\cos(\theta_1 - \theta_2) + 
		\eta_2\left| \psi_1 \right|^2\left| \psi_2 \right|^2
\end{align*} % (A1)

Давайте определим обозначенные величины
\begin{align*}
	\tilde{F} = \frac{4\pi}{\hbar^2 c^2}F \\
	\tilde{A} = -\frac{A}{\hbar c} \\
	\tilde{\psi}_a = \sqrt{\frac{4\pi}{m_a c^2}}\psi_a \\
	\tilde{\nu} = \frac{m_1m_2}\nu \\
	\tilde{\alpha}_a = \frac{m_a}{\hbar^2}\alpha_a \\
	\tilde{\beta}_a = \frac{m^2_a c^2}{4\pi\hbar^2}\beta_a \\
	\tilde{\eta}_1 = \frac{\sqrt{m_1 m_2}}{\hbar^2}\eta_1 \\
	\tilde{\eta}_2 = \frac{m_1 m_2 c^2}{4\pi\hbar^2}\eta_2
\end{align*} %(A2)

Тогда 
\begin{align*}
	\tilde{F} = \frac{1}{2}\left| 
		\left( \nabla + ie*\tilde{A} \right)\tilde{\psi}_2 \right|^2 + 
		\frac{1}{2}\left| 
		\left( \nabla + ie*\tilde{A} \right)\tilde{\psi}_2 \right|^2 - 
		\tilde{\nu} Re\left\{ 
		\left( \nabla + ie*\tilde{A} \right)\tilde{\psi}_1 \cdot 
		\left( \nabla + ie*\tilde{A} \right)\tilde{\psi}*_2 \right\} + \\
		\frac{1}{2}\left| \nabla\times\tilde{A} \right|^2 - 
		\tilde{\alpha}_1\left| \tilde{\psi}_1 \right|^2 + 
		\frac{\tilde{\beta}_1}{2}\left| \tilde{\psi}_1 \right|^2 + 
		\tilde{\alpha}_2\left| \tilde{\psi}_2 \right|^2 + \\
		\frac{\tilde{\beta}_2}{2}\left| \tilde{\psi}_2 \right|^2 + 
		\tilde{\eta}_1 \left| \tilde{\psi}_1 \right|
		\left| \tilde{\psi}_2 \right|\cos(\theta_1 - \theta_2) + 
		\tilde{\eta}_2 \left| \tilde{\psi}_1 \right|^2 
		\left| \tilde{\psi}_2 \right|^2
\end{align*} %(A3)

которая, опустив знак тильды, совпадает с представленным \eqref{eq:2} в 
данной статье. На протяжении всей статьи, предполагается, что группа(!) 1 
активна, то есть, \( \alpha_1 < 0 \). Перепишем уравнение \eqref{eq:2} для 
\( F \), так, чтобы \( \alpha_1 \) была нормирована на \( -1 \) и 
\( \beta_1 \) нормирована на \( 1 \). Это может быть достигнуто следующим 
образом. Напомним (см. раздел III A), что в отсутствии межзонной связи (т.е. 
когда \( \eta_1 = \eta_2 = \nu = 0 \)) condensate 1 имеет затухание линейного 
масштаба \( 1/\hat{\mu}_1 = (-4\alpha_1)^{-1/2} \). Эта scale обычно 
определяется длиной когерентности
\begin{align*}
	\hat{\xi}_1 = \frac{\sqrt{2}}{\hat{\mu}_1} = \frac{1}{\sqrt{-2\alpha_1}}
\end{align*} %(A4)

Ещё раз подчеркнём, что в присутствии межзонной связи, \( \hat{\xi}_1 \) вне 
длины когерентности condensate 1. Таково определение параметра с шляпой, 
напоминая, что это оригинальная длина когерентности только в несвязанном 
случае. Напомним также, что vacuum density of condensate в несвязанной модели 
\begin{align*}
	\hat{u}_1 = \sqrt{\frac{-\alpha_1}{\beta_1}}
\end{align*} %(A5)

Наша второе перемасштабирование \( \sqrt{2}\hat{\xi}_1 \) как единицы длины 
и \( \hat{u}_1 \) как единицы condensate density (вместе с компенсирующим 
масштабированием \( F \), \( e* \) и \( A \)). Конкретно говоря
\begin{align*}
	\bar{r} = \frac{r}{\sqrt{2}\hat{\xi}_1} = \sqrt{-\alpha_1 r} \\
	\bar{F} = \frac{2\har{\xi}^2_1}{\hat{u}^4_1}F = 
		\frac{\beta^2_1}{-\alpha^3_1}F \\
	\bar{\psi}_a = \frac{\psi_a}{\hat{u}_1} = 
		\sqrt{\frac{\beta_1}{-\alpha_1}}\psi_a \\
	\bar{A} = \frac{A}{\hat{u}_1} \\
	\bar{e} = \frac{1}{\sqrt{2}}\hat{u}_1\hat{\xi}_1 e* = 
		\frac{e*}{\sqrt{\beta_1}} \\
	\bar{\alpha}_2 = 2\hat{\xi}^2_1 \alpha_2 = \frac{\alpha_2}{-\alpha_1} \\
	\bar{\eta}_1 = 2\hat{\xi}^2_1 \eta_1 = \frac{\eta_1}{-\alpha_1} \\
	\bar{\eta}_2 = 2\hat{\xi}^2_1 \hat{u}^2_1 \eta_2 = 
		\frac{\eta_2}{\beta_1} \\
	\bar{\nu} = \nu
\end{align*} %(A6)

Подставляя их в \eqref{eq:2} получаем
\begin{align*}
	\bar{F} = \frac{1}{2}\left| 
		\left( \hat{\nabla} + i\hat{e}\hat{A} \right)\hat{\psi}_1 \right|^2 + 
		\frac{1}{2}\left| 
		\left( \hat{\nabla} + i\hat{e}\hat{A} \right)\hat{\psi}_2 \right|^2 -
		\nu Re\left\{ \left( \hat{\nabla} + i\hat{e}\hat{A} \right)
		\hat{\psi}_1 \cdot \left( \hat{\nabla} - i\hat{e}\hat{A} \right)
		\hat{\psi}*_2 \right\} + \\
		+ \frac{1}{2}\left| \hat{\nabla}\times\hat{A} \right|^2 - 
		\left|\hat{\psi}_1\right|^2 + 
		\frac{1}{2}\left|\hat{\psi}_1 \right|^2 - 
		\hat{\alpha_2}\left| \hat{\psi}_2 \right|^2 + 
		\frac{\hat{\beta}_2}{2}\left| \hat{\psi}_2 \right|^2 - \\
		- \hat{\eta}_1\left|\hat{\psi}_1\right|\left|\hat{\psi}_2\right|
		\cos(\theta_1 - \theta_2) + \hat{\eta}_2\left|\hat{\psi}_1\right|^2 
		\left|\hat{\psi}_2\right|^2
\end{align*} %(A7)

Это (опуская верхнюю черту) и есть энергия ГЛ которая 
используется в разделе IV для целей численного моделирования. 

Окончательно отметим, что однокомпонентная модель ГЛ получается 
из \eqref{eq:A7} определяя \( \psi_2 \equiv 0 \), глубину проникновения 
\( \lambda = 1/\mu_a = 1/e \) и длину когерентности 
\( \xi = 1/\sqrt{2} \), и, следовательно, параметр теории ГЛ
\begin{align*}
	\kappa = \lambda/\xi = \frac{\sqrt{2}}{e}
\end{align*} %(A8)

Так, в параметрах используемых в разделе IV, можно считать \( e \) в виде 
обратного параметра теории ГЛ для однозонной модели. Значение \( e \) 
соответствующий предельной однокомпонентной теории Богомольного c 
\( e_c = 2 \) в этой интерпретации. \newpage
    \chapter{Спектр \texorpdfstring{$ \tilde{\h} $}{H}}
\label{ch:B}
 \newpage
    \chapter{Расчет дальнодействующих внутривихревых сил из асимптотики линейного
  поля}
\label{ch:C}
 \newpage
  \end{appendix}
  \begin{thebibliography}{9}
  \addcontentsline{toc}{chapter}{Список литературы}
  \bibitem{1} 
\end{thebibliography}

\end{document}
