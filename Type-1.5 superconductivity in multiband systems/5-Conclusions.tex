\chapter{Выводы}
\label{ch:5}

В этой статье мы представили аналитическое и численное исследование о 
появлении сверхпроводимости типа 1.5 в случае двух зон с различными видами 
существенных межзонных соединений. Во всех случаях, которые мы рассматривали 
мы продемонстрировали, что система обладает тремя основными масштабами длин: 
первая \( 1/\mu_A \) связана с Лондоновской глубиной 
проникновения магнитного поля, в то время как остальные две \( 1/\mu_{1,2} \) 
связаны с характеристической масштабов длин ответственные за изменением 
глубины проникновения поля. В пределе двух конденсатов(?) связанных только с 
электромагнитными масштабными длинами \( 1/\mu_{1,2} \) с независимыми  
длинами когерентности двух полей. Однако мы показали, что введение ненулевой 
Джозефсоновской и density-density связи делают напряжённости полей спадающими 
по экспоненциальному закону при очень больших расстояниях от ядра, в то же 
время система всё ещё обладает двумя основными length scale, которые связаны 
с линейной комбинации напряженности полей повернутыми на <<угол смешивания>>. 
Третья основная масштабная длина в этом режиме является Лондоновской глубиной 
проникновения, и таким образом, двухзонная система со связью позволяет точно 
определить поведения типа 1.5. Далее мы изучали влияние смешанных градиентных 
членов и показали, как тип 1.5 описывается в этом случае. Мы показали, что в 
случае значительного смешанного градиента связи из определения трёх основных 
масштабных длин требуется дополнительные условия, поскольку он даёт режим 
смешивания, который не может быть описан одним углов смешивания. Важно 
отметить нами показанное, что смешанный градиент спаривания может увеличить 
несоответствие характерных масштабных длин изменения плотности. Можно провести 
аналогию между этим механизмом и механизмом качелей в физике нейтрино. Во 
второй части статьи мы представили сравнительное численное исследование 
вихрей типа 1.5 в различных режимах с различными межкомпонентными связями. 
Результаты были продемонстрированы в структуры двухкомпонентной модели 
Гинзбурга-Ландау с локальной электродинамикой. Однако мы ожидаем, что 
описание поведения типа 1.5 подобно присутствию в низких температурных 
режимах и в двухкомпонентных моделях с нелокальной электродинамикой.   

Понятие сверхпроводимости типа 1.5 можно непосредственно обобщить и в 
\( N \)-компонентном случае. Там может иметь место система с  
характерными масштабными длинами 
\( 
	\psi_1, \ldots, \psi_k < \lambda < \psi_{k+1}, \ldots, \psi_N
\) 
и есть термодинамически устойчивые вихри с немонотонным воздействием. 

Кроме многозонных сверхпроводников и сопутствующих электронных и ядерных 
сверхпроводников наша модель может быть реализована в искусственно созданных 
слоистых структурах из материалов типа-I и типа-II, где можно контролировать 
и настраивать межкомпонентную Джозефсоновскую связь.

\emph{Добавленное примечение:} После завершения этой работы, она была проверена в 
детальном расчёте, который не предполагает \( (1-T/T_C \) расширения, что 
модель Гинзбурга-Ландау \eqref{eq:2} довольно точно описывает физику вихрей 
двухзонных сверхпроводников в достаточно широком диапазоне параметров и 
температур\cite{bib:25}.